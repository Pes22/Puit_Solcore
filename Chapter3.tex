\chapter{النتائج و المناقشة } % Main chapter title

\label{Chapter3} % Change X to a consecutive number; for referencing this chapter elsewhere, use \ref{ChapterX}

%----------------------------------------------------------------------------------------
%	SECTION 1
%----------------------------------------------------------------------------------------
\section{ تمهيد}
أصبح نيتريد الغاليوم $ (GaN) $ واحدًا من أكثر المواد الواعدة ذات فجوة واسعة النطاق $ (3.4 فولت)  $و من مواد أشباه الموصلات المباشرة للاستخدام في ترانزستورات عالية الطاقة وعالية التردد ،  والصمامات الثنائية الباعثة للضوء الأزرق $ (LED)  $، وثنائيات الليزر $ (LDs)  $....،و في السنوات الأخيرة زاد الاهتمام بـ $ GaN  $المخدر بـ  شوائب الإربيوم ، بسبب الاحتمالية تطبيق هذه المادة في الإلكترونيات الضوئية.
ففي هذا الفصل قمنا بإستغلال هذه الدراسة في إنشاء هيكل لخلية شمسية  مصنعة من $ ErGaN / ErN  $ تحتوي على عدة أبار كمومية بإستخدام برنامح ال $ Solcore  $لإستخراج مقادير هذه الخلية فقد تطرقنا بداية إلى إعطاء لمحة حول المواد المستعملة ثم وصف لهيكل خليتنا الشمسية وبعدها تطرقنا إلى عرض النتائج المتحصل عليها 
\section{   لمحة عن المواد المستعملة في الدراسة  }
\subsection{  نتريد الغاليوم المطعم بالإربيوم ErGaN   }
العناصر المسماة "بعناصر الأرض النادرة" $ (RE)  $لها غلاف داخلي مملوء جزئيًا (4f) محمي من المناطق المحيطة به بواسطة مدارات خارجية ممتلئة بالكامل $ (5s و 5p)  $، ينتج عن التحولات داخل الغلاف 4f انبعاثات بصرية حادة جدًا بأطوال موجية من الأشعة فوق البنفسجية إلى الأشعة تحت الحمراء ، يتم تحديد الأطوال الموجية لخطوط الانبعاث هذه بواسطة  إنتقال الطاقة من   4f ، المادة المضيفة لها تأثير قوي جدًا على احتمالية الانتقال الإشعاعي ، وبعبارة أخرى على كثافة الانبعاث الضوئي. بشكل عام ، أظهرت أشباه الموصلات التقليدية  $ (Si ، GaAs ... إلخ)  $ إنبعاث ضوئي محدود في درجة حرارة الغرفة ،فإن أشباه الموصلات ذات فجوة الحزمة العريضة  هي مضيفات    ممتازة  لعناصر الأرض النادرة ومن المضيفات التي سنتطرق لها هي نتريد الغاليوم $ GaN  $ و المادة المضافة هي عنصر الأرض النادر إربيوم Erbuim
للحصول على شبه الناقل ErGaN  
\begin{figure}[h!]
	\centering
	\includegraphics[width=0.7\linewidth]{"Fig/Fig_III/ rare earth"}
	\caption{عناصر الأرض النادرة في الجدول الدوري }
	\label{fig:-rare-earth}
\end{figure}
\FloatBarrier
\subsubsection{ نتريد الغاليوم (GaN)}
هو مادة شبه موصلة تنتمي إلى عائلة أشباه الموصلات III-V يختلف هيكلها البلوري بين سداسي ومكعبي   ، يمكن تطعيم GaN بالسيليكون أو بالأكسجين للحصول على شبه موصل من نوع $ n-type  $وتطعيمه بالمغنيزيوم للحصول على $ p-type  $، بالإضافة إلى أن الجاليوم وسبائك الغاليوم تمتلك خصائص كهروضغطية فعندما يتم تطبيق مجال كهربائي خارجي على مواد كهروضغطية تتشوه ميكانيكيا ، وعلى عكس فهذه المواد  تولد   مجالا كهربائيا إستجابة لإجهاد ميكانيكي مطبق عليها ، فعند زراعة مادتين بثابث شبيكة مختلف فإن الطبقة أعلاه سوف تتمدد أوتتقلص بحيث يتطابق ثابت الشبكة مع الطبقة  التي تنمو عليها

\begin{table}
	\centering
	\begin{tabular}{lll}
		\hline\noalign{\smallskip}
		الكمية              &  الوحدة   &~~  القيمة   \\
		\noalign{\smallskip}\hline\noalign{\smallskip}
		فجوة النطاق            &  $ ev $  &~~   3.4 \\
		حركية الإلكترون     &  $ cm^2/Vs  $   &~~ 1800    \\
		حركية الفجوات       &      $  cm^2/Vs$ &~~  30    \\
		الموصيلية الحرارية     &    $ w/cm K $    &~~   1.3\\
		الكثلة الفعالة للإلكترون  &    &~~       $ 0.2 m_e  $\\
		ثابت العزل النسبي   &    &~~  8.9\\
		معامل الإمتصاص    &  $ cm^{-1} $  &~~    $  10^5 $    \\
		electron affinity    &  $  V$  &~~    $   4.1 $    \\
		
		
		
		\noalign{\smallskip}\hline
	\end{tabular}
	\caption{     يبين بعض خصائص   لنتريد الغاليوم  }
	\label{tab:1}
	
\end{table}
\FloatBarrier
\subsubsection{ الإربيوم Er}
الإربيوم  Er 68 هو الأيون الحادي عشر في سلسلة اللانثانيدات  لعناصر الأرض النادرة هيكله الإلكتروني هو :       
$ [Xe], 4f^{12},6^{2} $
الإربيوم النقي معدن ،عندما يتم تصنيفها على أنها شوائب في مضيف عازل ، تأخذ Er عادة حالة الشحنة ثلاثية التكافؤ  مع التوزيع الإلكتروني:
$ [Xe], 4f ^{11} $
\begin{table}
	\centering
	\begin{tabular}{lll}
		\hline\noalign{\smallskip}
		الخاصية   & الوحدة  & قيمة \\
		\noalign{\smallskip}\hline\noalign{\smallskip}
		الوزن الذري  &   &   ~~   86 ~  \\
		الكثافة  &   $ kg /m  $    & ~~  9066  \\
		المقاومة الكهربائية &$ \Omega m $ &~~ 860e-8~   \\
		الموصيلية الحرارية  &$ w/m.k $ &~~ 15  ~   \\
		الموصيلية  النوعية& $ \Omega m $ &~~ $  8.6e-7 $  ~   \\
		نقطة الغليان   &$ C $ &~~  2868  ~   \\
		نقطة الإنصهار  & $ C $ &~~   1497   ~   \\
		ثابت الشبيكة  & $ pm $  &~~    355.88   ~   \\
		
		
		\noalign{\smallskip}\hline
	\end{tabular}
	\caption{ بطاقة تقنية للإربيوم Erbuim  }
	\label{tab:1}
\end{table}
\FloatBarrier
\subsection{ نتريد الأربيوم  ErN  }
نيتريد الإربيوم $ (ErN)  $هو أحادي نيتريد فلز أرضي نادر له خصائص إلكترونية ومغناطيسية وبصرية مرغوبة ، معدن الإربيوم هو مادة الخام بعد التسامي  تتشكل بلورات ErN
يمكن دمج $ ErN  $في أشباه الموصلات $ III-nitride $ لتطوير مواد وظيفية جديدة للأجهزة الإلكترونية الضوئية. يحتوي $ ErN  $على فجوة طاقة صغيرة غير مباشرة تبلغ حوالي 0.2 فولت مع حد أدنى لنطاق التوصيل عند النقطة   X  من منطقة Brillouin ونطاق تكافؤ أقصى عند النقطة $\Gamma$ . أصغر فجوة طاقة مباشرة متوقعة حوالي 1 فولت ، مع نطاقي تكافؤ عند النقطة  X  
الشكل . \ref{fig:ern}
يوضح ذلك 
 \begin{figure}[h]
 	\centering
 	\includegraphics[width=0.7\linewidth]{"Fig/Fig_III/ ErN_structure"}
 	\caption{يبين بنية النطاق الإلكتروني لنتريد الإربيوم}
 	\label{fig:ern}
 \end{figure}
 

\begin{table}
	\centering
	\begin{tabular}{ll}
		\hline\noalign{\smallskip}
		إسم الخاصية  & قيمة الخاصية    \\
		\noalign{\smallskip}\hline\noalign{\smallskip}
		الوزن الجزيئي  &  181.27 \\
		ثابت الشبكة  & 4.789 \\
		حركية الإلكترون & 0.001   \\
		حركية الفجوات  & 0.001 -\\
		الكثلة الفعالة للفجوات الثقيلة &  0.61\\
		الكثلة الفعالة للفجوات الخفيفة &  0.18\\
		نسبة الصلابة على معامل الكسر &  0.61\\
		\noalign{\smallskip}\hline
	\end{tabular}
	\caption{  يبين بعض الخصائص لنتريد إربيوم  }
	\label{tab:1}
\end{table}
\FloatBarrier
\section{  وصف للخليةالشمسية المستعملة ونبدة عن البرنامج التعليمي المطبق }
سنقوم في هذا البرنامج بإنشاء خلية شمسية وحساب خصائصها باستخدام     $ Solcore $  \\
المخرجات :
\begin{itemize}
	\item خصائص الإمتصاص للبئر الكمي 
	\item  كفاءة الكم الخارجيةEQE
	\item  كفاءتها $ (Eff) $ ، تيار الدائرة القصيرة $ (I_sc) $ ، جهد الدائرة المفتوحة $ (V_oc)  $وعامل الإمتلاء  $ (FF) $ كدالة للتركيز
\end{itemize}
\subsection{ البئر الكمي ErGaN/ErN } 
البئر الكمي هو بنية غير متجانسة من أشباه الموصلات مصنعة لتنفيذ التأثيرات الكمومية في التطبيقات الإلكترونية والفوتونية. وهي عادةً طبقة رقيقة جدًا من أشباه الموصلات ذات فجوة الحزمة الضيقة المحصورة بين طبقتين من أشباه الموصلات ذات فجوة الحزمة الأكبر،هنا تكون الثقوب حرة في التحرك في الاتجاه العمودي لاتجاه نمو البلورات وليس في اتجاه نمو البلورات ، وبالتالي فهي محصورة. سيؤدي حل معادلة شرودنجر لإمكانية محدودة إلى إنتاج قيم مستويات الطاقة داخل البئر.يتكون البئر الكمي المتعدد  (MQW) من طبقات متناوبة من $ ErGaN $و $  ErN $ ذات عروض معينة .
\subsubsection{ تحديد مواد البئر الكمي:}
أول شيء قمنا  بإضافة مادتي ErGaN و ErN كمواد جديدة  في Solcore لأن البرنامح لايحتويهما بنفس الخطوات المذكورة في  الفصل.\ref{Chapter2}
\\
قمنا بتعيين وتحديد  مواد للبئر الكمي : نتريد الغاليوم الإربيوم$  ErGaN $ ، نتريد الإربيوم $ ErN $ \\
يبلغ عرض البئر الكمي  3 نانومتر وسماكة الحاجز   5 نانومتر  ، جهازنا يحتوي على أربعة أبار كمومية : حيث تمثل طبقة ErN الحاجز وطبقة ErGaN البئر \\
يجب كتابة السطر التالي في التعليمات البرمجية لحل  الخصائص الكمومية لـ QW ، مع ترك القيم الافتراضية لجميع المعلمات: 
\begin{flushleft}
	$ QW\_list = QW.GetEffectiveQW(wavelengths=wl) $ 
\end{flushleft}
حيث تقوم الوحدة $ GetEffectiveQW $  بإستخدام لأدوات المساعدة داخل وحدة ميكانيك الكم  لحساب بنية النطاق لـ $ QWs $ ، ومعامل الامتصاص الخاص بها ، وأخيراً ، ستحسب فجوة النطاق الفعالة ، وكثافة الحالات ، وما إلى ذلك التي سيستخدمها محلل $ PDD $. 
\subsection{ تحديد الوصلة junction }

نحتاج إلى إعطاء كل الطبقات والمواد التي تتكون منها الوصلة  $ junciton $ ، بالطريقة نفسها التي فعلناها مع $ QWs $. شيء واحد يجب ملاحظته هو أنه إذا لم يتمكن $ Solcore  $من العثور على خاصية تحتاجها لحل معادلات $ PDD  $، فستأخذ الخاصية المقابلة لـ $ GaAs  $كقيمة افتراضية. لذلك يجب التأكد من تقديم جميع القيم المطلوبة.
\begin{itemize}
	\item  \textbf{ الطبقة n} :
	من  $ ErGaN  $نسبة الشوائب   المانحة فيها $ 1e8 $  عرضها $ 10e-9 $
	\item \textbf{الطبقة P} :
	من  $ ErGaN $ نسبة الشوائب المستقبلة فيها   $ 3e17 $ عرضها$  50e-9 $
\end{itemize}
\subsubsection{ تحديد الطلاء AR }
سيقلل طلاء $ AR  $من انعكاس السطح الأمامي ، وبالتالي يزيد من التيار الضوئي للخلية الشمسية. نحن نستخدم طلاء بسيط مزدوج الطبقة مصنوع من $ MgF2 $ و  $ ZnScub $. كلتا المادتين متاحتان في قاعدة بيانات $ SOPRA  $للثوابت البصرية
\subsection{  تكوين الخلية  }
مع تحديد جميع المواد والهياكل ، نحتاج فقط إلى تجميع كل شيء معًا بما في ذلك بما في ذلك الطلاءات والوصلة الخاصة بنا 
 \section{ دراسة النتائج وتحليلها }
 \subsection{ تأثير شدة الإضاءة على  الخلية الكهروضوئية   }
 \subsubsection{ تأثير  شدة الإضاءة على كثافة التيار}
 يمثل المنحنى  \ref{fig:current} تغير كثافة التيار بدلالة شدة الإضاءة ،حيث نلاحظ أن كثافة التيار تزداد وفق خط مستقيم بتزايد شدة الإضاءة ،أي هناك تناسب طردي بين كثافة التيار و شدة الإضاءة وقيمة التيار $ I_{sc} $ تتخطى قيمة  10e5
 \begin{figure}[h!]
 	\centering
 	\includegraphics[width=0.7\linewidth, height=0.3\textheight]{Fig/Fig_III/04-1}
 	\caption{كثافة التيار بدلالة شدة الإضاءة}
 	\label{fig:current}
 \end{figure}
 \FloatBarrier
   حيث يمكن أن نستخلص أنه كلما زادت الإضاءة تزداد معها شدة التيار ،هذه الزيادة تسمح للخلية الشمسية بإنتاج طاقة كهربائية أكبر
 
 \subsubsection{ تأثير   شدة  الإضاءة على  جهد الدارة}
 عند سقوط أشعة الشمس على سطح  خلية شمسية تنتج جهدا في الظروف القياسية ويعرف هذا الجهد  بجهد الدارة المفتوحة $  V_{oc} $ ويعبر عنه عندما تكون الخلية غير موصلة بالأحمال الكهربائية وتكون قيمة التيار الكهربائي مساوية للصفر ،
 يمثل المنحنى تغيرات جهد الدارة المفتوحة بدلالة شدة الإضاءة ، حيث نلاحظ أن هناك علاقة طردية بين جهد الدارة المفتوحة وشدة الإضاءة وتصل إلى   $ 0.225 $ 
 \begin{figure}[h!]
 	\centering
 	\includegraphics[width=0.7\linewidth, height=0.3\textheight]{Fig/Fig_III/06-1}
 	\caption{جهد الدارة المفتوحة بدلالة شدة الإضاءة}
 	\label{fig:06-1}
 \end{figure}
 
 \FloatBarrier
 
 \subsubsection{    تأثير   شدة الإضاءة على   معامل الإمتلاء}
 يوضح المنحنى الأتي .\ref{fig:05-1} تغيرات معامل الإمتلاء $ FF $ مع تركيز إشعاع الشمس في نطاق $ 10 -10e3  $ لخلية البئر الكمي $ ErGaN  $و $ ErN  $، حيث نلاحظ إرتفاع لمعامل الإمتلاء بزيادة تركيز الشمس ليصل إلى قيمة أكثر من$ 65\% $ \\
 حسب المعدلة .\ref{FF_eq}نجد أن $ FF $ له علاقة ب " $  V_m $ , $ I_m $ ، $ V_{oc} $ ، $  I_{cc}$ " و منه نستطيع القول 
 أن الإستطاعة المأخودة من الخلية الشمسية تتزايد بتزايد شدة الإضاءة ، وهذا التزايد سسببه الزيادة في الجهد $ V_{oc} $في الدارة المفتوحة وذلك بالتوافق مع الزيادة الخطية للتيار الضوئي بزيادة الشدة الضوئية الراجع إلى  الزيادة في تركيز  حوامل الشحنة الأقلية عند شدات إضاءة مرتفعة  \\
 وبالتالي قيم معامل الإمتلاء مرتفعة وهذا يعود إلى إرتفاع في ثابت الإمتصاص الضوئي لمادة ErGaN 
  
 \begin{figure}[h!]
 	\centering
 	\includegraphics[width=0.7\linewidth, height=0.3\textheight]{Fig/Fig_III/05-1}
 	\caption{معامل الإمتلاء بدلالة شدة الإضاءة}
 	\label{fig:05-1}
 \end{figure}
 \FloatBarrier
 \subsubsection{ تأثير شدة الإضاءة  على  المردودية}
 يوضح المنحنى المقابل كفاءة التحويل الكهروضوئي (المردودية) كدالة لشدة الإضاءة ،ونلاحظ أن القيمة القطوى للمردودية تكون أكثر بقليل  من $ 2.5\%   $  \\
 الإزدياد في المردودية يتناسب مع الزيادة في شدة الإضاءة وهذه النتائج هي تحصيل حاصل لقيم المتغيرات "  $ I_{so} $،  $ V_{co} $
 ، $ FF $ " 
 حسب العلاقة .\ref{n_eq} 
 \begin{figure}[h!]
 	\centering
 	\includegraphics[width=0.7\linewidth, height=0.3\textheight]{Fig/Fig_III/03-1}
 	\caption{المردودية بدلالة  شدة الإضاءة}
 	\label{fig:03-1}
 \end{figure}
 \FloatBarrier
 
 \subsection{ تأثر منحنى تيار -الجهد بالإشعاعية }
 إنتاجية الخلية الكهروضوئية تعتمد بشكل مباشر على كمية الإشعاع الساقط على سطحها . كلما إرتفعت كمية الإشعاع الشمسي سوف ترتفع قيمة التيار الكهربائي الخارج من الخلية الكهروضوئية بسبب زيادة مستوى التأثير الكهروضوئي ، في المقابل تتغير الفولطية بشكل ضئيل بتغير الإشعاعية ، وهذا يعني أنه بمجرد سقوط الإشعاع الشمسي  على الألواح الكهروضئية  الصباح يرتفع الجهد إلى قيمة قريبة من جهد الدارة المفتوحة وبغض النظر عن التغير في الإشعاع الشمسي سوف تتغير قيمة  الفولتية بشكل بسيط . \\
 تعرف هذه الخلية الشمسية بأنها تقوم بإنتاج فولتية مقدارها أقل من  $ 0.5  $فولط وتيار يتناسب مع شدة الإشعاع الشمسي مقدره $ 1 $ أمبير في حالة إشعاع شمسي قدره $ 1000W/m^2 $ \\
 لنستنتج أن الإشعاع الشمسي لا يؤثر بشكل كبير على الجهد عكس التيار 
 \begin{figure}[h!] 
 	\centering
 	\includegraphics[width=0.7\linewidth]{Fig/Fig_III/Figure_1}
 	\caption{ تغيير قيم الجهد والتيار تبعا للإشعاع الشمسي}
 	\label{fig:figure1}
 \end{figure}
 \FloatBarrier
 \subsection{ كفاءة الكم الخارجية EQE }
 تشير قيمة كفاءة الكم في الخلية الشمسية الى مقدار التيار الذي ستنتجه الخلية عند تشعيعها بفوتونات ذات طول موجي معين . أما كفاءة الكم الخارجية  EQE هو نسبة عدد حاملات الشحنة التي تم جمعها بواسطة الخلية الشمسية إلى عدد الفوتونات الساقطة 
 \begin{equation}
 	 EQE=\frac{Electrons/sec}{Incident~ photons/sec}
 \end{equation}
 \begin{figure}[h!]
 	\centering
 	\includegraphics[width=0.7\linewidth, height=0.3\textheight]{Fig/Fig_III/02-1}
 	\caption{تغيرات كفاءة الكم الخارجية $ EQE  $والإمتصاص مع الطول الموجي }
 	\label{fig:02-1}
 \end{figure}
 \FloatBarrier
 يوضح الشكل  .\ref{fig:02-1} الكفاءات التي يمكن تحقيقها مع الإمتصاص حيث نلاحظ من المنحنى أن الامتصاص في الاطوال الموجية القصيرة يبدأ من قيمة معتبرة $   0.8$ ثم يتناقص مع زيادة الطول الموجي في حين كفاءة الكم الخارجية تبدأ من قيمة أقل من $ 0.6  $وتبقى في تذبذب حتى طول موجي $ 500  $فترتفع ارتفاعا اعظميا وهذا عكس الامتصاص الذي يبقى حتى طول موجي بقيمة $ 900  $ ثابتا بين حدود  $ 0.4 $ و $ 0.6  $ وذروته تبلغ تقريبا 1.0 بين طول موجي $ 900  $و $ 1000  $ثم يستمر في الانخفاض بعد هذا الطول الموجي فتتبعه كفاءة الكم الخارجية  
 \subsubsection{  تحليل  }
  من خلال المنحنى  .\ref{fig:02-1}
 \begin{itemize}
 	\item      

 تعتمد EQE على تجميع الشحنات و إمتصاص الضوء هناك فرق كبير بين المنحنين ولذلك نستطيع القول أن هناك ضياع في كمية الفوتونات التي تتسرب دون أن تمتص  أو الفوتونات العارضة التي تنعكس بعيدا عن سطح الخلية  
 \item
 بمجرد امتصاص الفوتون وتكوين زوج من الثقوب الإلكترونية ، ييتم فصل هذه الشحنات و إعادة تركيبها ، المادة "الجيدة" تتجنب إعادة تركيب الشحنة لأنه يساهم في انخفاض كفاءة الكم الخارجية   
 \item 
   نستطيع القول كذلك بأن الفوتونات الساقطة بها أكثر من ضعف طاقة فجوة النطاق ويمكنها إنشاء اثنين أو أكثر من أزواج الثقوب الإلكترونية لكل فوتون  ساقط 
    \end{itemize}
   
 \subsection{   حاملات الشحنة   } 
 
 \begin{figure}[h!]
 	\centering
 	\includegraphics[width=0.7\linewidth]{Fig/Fig_III/01-1}
 	\caption{حاملات الشحنة بدلالة الموضع }
 	\label{fig:01-1}
 \end{figure}
 \FloatBarrier
 من الشكل نلاحظ أن حاملات الأكثرية هي الإلكترونات بنسبة  10e10 في طول موجي أقل من 200  أما بالنسبة لتركيز الفجوات فيكون
 \section{ خاتمة}
    لقد أظهرت الآبار الكمية نتائج واعدة لحصاد الطاقة  ويرجع ذلك إلى حد كبير  إلى  قدرتها على التقاط العديد من الأطوال الموجية المختلفة للضوء حيث  يتم إمتصاص طاقة الفوتونات داخل $ pn $ الكمومية   فالمنطقة الجوهرية تحتوي على أكثر من بئر أدى إلى زيادة التيار الضوئي ، مما يؤدي إلى زيادة الكفاءة على الخلايا التقليدية ،حيث يجب أن تؤخذ آليتان رئيسيتان في الاعتبار من أجل تحسين كفاءة الأجهزة الكهروضوئية: امتصاص الفوتون و توليد  الناقل ولقد أظهرنا أن وجود عدة $ MQW  $ يسمح بتحسين الامتصاص بالإضافة إلى طبيعةالمواد  المكونة للخلية 
 
 
 