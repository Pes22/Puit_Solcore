
\chapter{ محتويات برنامج المحاكاة solcore }
\label{Chapter2} 
%----------------------------------------------------------------------------------------

% Define some commands to keep the formatting separated from the content 
%----------------------------------------------------------------------------------------

\section{ تمهيد}
بحث العلماء عن وسيلة تساعدهم في معرفة نتائج مختلف التجارب العلمية التي يقومون بها قبل تنفيذها على أرض الواقع تجنبا لوقوع في مشاكل ومواقف هم في غنى عنها فظهر ما يدعى بمصطلح النمدجة والمحاكاة القائمة على إيجاد نماذج إفتراضية لمختلف الأنظمة و إجراء التجارب والعمليات عليها ومراقبة النتائج فإن كانت  وفق المتوقع والمرغوب طبقت على الأنظمة الحقيقية وإن لم تكن كذلك أجريت بعض التعديلات و تختلف عمليات المحاكاة بإختلاف برامجها وقد كان النصيب الأوفر للخلايا الشمسية في هذا  فقد تم تطوير العديد من البرامج  لمحاكاة الخلايا الشمسية ومن بين هذه البرامج Solcore  حيث سنتطرق في هذا الفصل إلي تقديم وصف عام وسطحي لكيفية تثبيت هذا البرنامج وبعض أساسياته بالإضافة إلى أهم الأدوات التي يستخدمها في دراسة الخلايا الشمسية وأشباه النواقل وحلها .	

\section{لغة البرمجة بايثون Python}
بايثون python هي لغة برمجة عالية  المستوى طورت من قبل Guido van Rossum  وقد إستقى هذه اللغة من عدة لغات سابقة من أمثال  ‫‪Icon‬‬ ‫و‬ ‫‪ABC‬‬ ‫و‬ ‫‪Modula-3‬‬ ‫ و‬ ‫‪C‬‬ ‫‪++‬‬ ‫و‬ ‫‪C‬‬ ،‫  ونشر ت أول مرة في عام  1991 ، هي لغة متعددة الإستعمالات ل كتابة السكريبتات ، الألعاب ، بناء البرامج .....الخ ،و هي من أكثر الغات شيوعا لتطوير البرامج مفتوحة المصدر  ، بعض مميزات هذه اللغة :
\begin{itemize}
	\item   يسهل تعلم لغة البايثون إذ تتألف من عدد قليل من الكلمات المفتاحية وتتميز بصياغة بسيطة ومحددة 
	\item   شيفرة لغة البايثون واضحة ومنظمة وسهلة القراءة 
	\item   تحتوي مكتبة بايثون القياسية على عدد كبير من الحزم المحمولة التي تتوافق مع أنظمة يونكس وويندوز وماك
	\item   تدعم بايثون الوضع التفاعلي مما يتيح إمكانية تنفيذ الشيفرات مباشرة على سطر الأوامر التفاعلي 
	\item   بايثون مناسب للبرامج الكبيرة والمعقدة
	[ 24،25 ]
\end{itemize}
\begin{figure}
	\centering
	\includegraphics[width=0.7\linewidth, height=0.22\textheight]{Fig/Fig_II/PYTHON}
	\caption{شعار لغة البرمجة البايثون }
	\label{fig:python}
\end{figure}
\FloatBarrier
\section{ تعريف وتثبيت برنامج Solcore}

\subsection{ تعريف :}
هو برنامج محاكاة للخلايا الشمسية وأشباه النواقل مكتوب بلغة البايثون فقد تطور مع مرور الوقت بإعتباره حلا كاملا لأشباه الموصلات قادر على نمدجة الخصائص الضوئية والكهربائية لمجموعة واسعة من الخلايا الشمسيىة من أجهزة الأبار الكمومية إلى الخلايا الشمسية متعددة الوصلات ، هو نتيجة سنوات عديدة من التطوير من قبل العديد من الأشخاص المساهمين : $ D. Alonso ~Alvarez $ .$ T.Wilson $ .$ P.Pearce $ .$ M. Fuhrer  $.$ D.Fzrrell $ . $ N. EKins-Daukes $
يستخدم solcore نهجا للحصول على معلمات المواد من مصادر مختلفة منها: 
\begin{flushleft}
	
	$ article~by ~Vurgaftman ~on~ band~ parameters~ for~ III-V ~semiconductors $
	\\
	$ Handbook~ Series ~on ~semiconductor~ Parameters ~by~ Levinshtein ~ et~ al $
	\\
	$ sotoodeh $ 
	\\
	$ Adachi  $  
	
\end{flushleft}

\begin{figure}[h]
	\centering
	\includegraphics[width=0.5\linewidth]{Fig/Fig_II/SOLCORE}
	\caption{شعار برنامج solcore}
	\label{fig:solcore}
\end{figure}
\FloatBarrier
\subsection{ الثتبيت والتكوين }
يمكنك تجربة برنامج السولكور على الحاسوب دون تثبيته وذلك اون لاين ، ولتثبيته  على الحاسوب نحتاج للعناصر التالية:
\begin{enumerate}
	\item بايثون النسخة الاخيرة $ 3.7 $
	\item pip 
	\item setuptools 
	\item  مترجم fortran مناسب( فقط من أجل PDD ) 
	\item حزمة S4 ( مطلوب لوظيفة RCWA )
\end{enumerate}
ويكون تثبيت هذه العناصر عن طريق الأوامر التالية :\\    
\begin{flushleft}
	$ pip ~ install ~-U ~pip~ setuptools~ wheel $\\
	$ pip~ install ~solcore $
	$ pip ~install ~--no-deps~ --force-reinstall --install-option="--with\_pdd"~ solcore $
	
\end{flushleft}
\textbf{بدء العمل في  Solcore :} 
يمكنك الأن البدء في العمل على برنامج سولكور في حاسوبك الخاص بعد تثبيثه حيث سيتم انشاء مجلد مخفي في دليل المستخدم الخاص بك سيحتوي هذا المجلد على التكوين المحلي وسيخزن المواد والمعلمات الأخرى . ويمكنك الاستعانة بوحدة $ config $ عبر الأمر التالي :\begin{flushleft}
	$ 
	from ~solcore ~import~ config $
	\\
	$ help~(config)
	$
\end{flushleft}
وحدة $ The~ config~ tools  $: تحتوي على جميع الوظائف التي تساعدك على إعداد وتكوين وتثبيت السولكور الخاص بك .  
\section{ الهياكل ودروس الدعم}
بينما يتعلق $ Solcore $بالفيزياء في الغالب فانه يحتاج أيضا إلى الكثير من الأدوات التي تستمر في العمل معا بشكل صحيح أو التي تدعم انشاء وادارة بنية الخلايا الشمسية . هنا يمكنك ان تجد بعص المعلومات حول هذه الأجزاء والقطع الاضافية التي يتكون منها : \\
\textbf{هيكل (Structure) :}
يحسب $ Solcore $ الخصائص الضوئية والكهربائية للخلايا الشمسية أو بعبارة أخرى مجموعة معينة من الطبقات المصنوعة من مواد مختلفة وتخدم أغراضا محددة . تحتوي وحدة الهيكل $ (Structure)  $على وحدات البناء الأساسية التي تسمح لك بانشاء هيكل خلية شمسية وحساب خصائصها \\
\textbf{وحدة الخلايا الشمسية $ (Solar Cells) $ :}
تضمن اللبنة الأساسية للخلايا الشمسية ذات المستوى الأعلى في وحدة $ Solar Cells $ \\
\textbf{تعقب العلوم  $ Science tracker $ :} 
برنامج $ Solcore  $هو عمل أصلي ، ولكن المعادلات التي ينفذها والبيانات التي يستخدمها تم نشرها في كثير من الأحيان. يسمح لك متتبع العلوم بتتبع تلك المراجع والتحقق من مصدرها وافتراضاته بنفسك. 
\section{ المواد والوحدات }
هذه الوحدات النمطية التي تتعامل مع خصائص المواد والوحدات جنبا إلى جنب مع وحدات الهيكل $ (Structure)  $فانها تشكل العمود الفقري لبرنامج $ Solcore $ 
\subsection{   قاعدة بيانات المعلمات }
تحتوي قاعدة بيانات المعلمات على الخصائص الأساسية للعديد من أشباه الموصلات منها السليكون والجرمانيوم والسبائك الثنائية والثلاتية III-V من بين هذه المعلمات فجوة نطاق الطاقة ،الكثل الفعالة للإلكترون والثقوب ،ثوابت الشبكة وثوابت المرونة..... \\
هنالك طريقة لاسترجاع المعلمات من قاعدة البيانات تكون باستخدام الوظيفة $ get\_parameter  $ مع لمدخلات المطلوبة 
\\
مثال : \\
$ get\_parameter("GaAsP", "band\_gap", P=0.45, T=300) $
\\
سيعيد فجوة النطاق لـ $ GaAsP  $ لتركيز الفوسفور بنسبة 45٪ عند درجة حرارة 300 كلفن ، أي ما يعادل 1.988 فولت. تستخدم هذه الطريقة البيانات
الموجودة فقط. هناك طريقة أخرى تتمثل في إنشاء كائن مادي يحتوي على جميع الخصائص الموجودة في قاعدة البيانات لتلك المادة ، بالإضافة إلى تلك المدرجة كمدخلات ، والتي ستتجاوز قيمة معلمة قاعدة البيانات ، إن وجدت. ينشئ المثال التالي كائن $ GaAs $ وكائن $ AlGaAs  $، باستخدام كتلة فعالة إلكترونية مخصصة في الأخير:
\\
\begin{flushleft}
	$ GaAs = material("GaAs")(T=300, Na=1e24) $ \\
	$ AlGaAs = material("AlGaAs")(T=300, Al=0.3, Nd=1e23, eff\_mass\_electron=0.1) $ 
\end{flushleft}
الان اي معلمة بما في ذلك المعلمات المخصصة هي سمات يمكن الوصول اليها بسهولة واستخدامها في اي مكان في البرنامج 
\subsection{  قاعدة بيانات الخصائص الضوئية : }
من أجل حساب ونمذجة الإستجابة الضوئية لهياكل الخلايا الشمسية وأنظمة المواد، يعد الوصول إلى مكتبة بيانات ضوئية ثابتة أمرا ضروريا لذلك 
يشتمل Solcore على مورد للبيانات الضوئية المتاح مجانا من قبل شركة $ Sopra ~SA  $. حيث توفر قاعدة البيانات هذه معامل الإنكسار n ومعامل   معامل الخمول K  لجميع الأطوال الموجية لعدة فئات من المواد منها :$ (II-VI)  $,$ (III -V)  $وبعض المعادن... ، في حال عدم توفر هذه البيانات يتم وضع  n=0 و K=1 لكافة الأطوال الموجية         
تسمح لنا هذه البيانات بحساب ونمدجة الإستجابة الضوئية لهياكل الخلايا الشمسية والمواد المتوفرة .
\subsection{ تحديد مواد جديدة   :  }
توجد طريقتان لاستخدام مواد غير مضمنة في قاعدة بيانات $ Solcore $ والتي يمكن توصيلها به وهي :
\begin{itemize}
	\item تنزيل واستخدام قاعدة البيانات من : $ Defining~ new  ~materials $
	\item  توفير بيانات ال n  و k  والمعلمات الاخرى لأمر $ create\_ new\_material $
\end{itemize}
من اجل التحكم في مكان حفظ المواد المخصصة نحتاج الى اخبار ال $ Solcore $ بمكان انشاء المواد الجديدة والبحث عنها عن طريق اضافة بعص الادخالات الى ملف تكوين المستخدم الخاص بك ( افتراصيا المجلد المعني يسمى $ solcore\_config.txt $ في الدليل الخاص بك ) وذلك بالطريقة التالية :
\begin{itemize}
	\item المسار الذي سيتم تنزيل قاعدة بيانات $ refractiveindex.info $ إليه يتم تعيينه ضمن [Other] مع الإشارة nk
	\item يتم تعيين المسار حيث سيتم حفظ بيانات n و k ضمن [Other] مع وضع علامة$  custom \_mats $
	\item يتم تعيين المسار حيث سيتم إنشاء الملف الذي يحتوي على معلمات المواد المخصصة ضمن [Parameters] مع علامة مخصصة
\end{itemize} 
اضافة مواد جديدة لقاعدة البيانات : 
\begin{flushleft}
	$ solcore.material\_system.create\_new\_material.create\_new\_material(mat\_name, n\_source, k\_source, parameter\_source=None, overwrite=False) $
\end{flushleft} 
تضيف هذه الوظيفة مادة جديدة إلى مجلد $ material\_data  $الخاص بـ $ Solcore  $، بحيث يمكن تسميتها كمواد مضمنة. يحتاج إلى اسم للمادة الجديدة ، وملفات المصدر لبيانات n و k والمعلمات الأخرى التي سيتم نسخها في مجلد $ material\_data  $/ $ Custom $ 
\\ 
المعلمات: 
\begin{itemize}
	\item mat\_name - اسم المادة الجديدة
	\item  n\_source - مسار قيم n (ملف txt ،$ n  $ الطول الموجي للعمود الأول بالمتر ، العمود الثاني)
	\item  k\_source - مسار قيم  k (ملف txt ، الطول الموجي للعمود الأول بالمتر ، العمود الثاني $ k $)
\end{itemize}
عند إضافة مادة جديدة إلى قاعدة البيانات ، سيتم إنشاء مجلد جديد لها كمجلد فرعي للمسار المحدد في التكوين (المستخدم أو الافتراضي) ضمن مسار $ custom\_mats $ سيتم نسخ ملفات البيانات n و k التي تقدمها إلى هذا المجلد (تتم إعادة تسميتها تلقائيًا). سيتم نسخ أي معلمات أخرى توفرها في الملف المحدد ضمن المسار custom   
\section{ الحلول الكمية Quantum Solver :}
إن بنية النطاق الإلكتروني لمواد أشباه الموصلات مسؤولة عن امتصاص الضوء وخصائص الانبعاث بالإضافة إلى العديد من خصائص النقل الخاصة بها ، والتي تعتمد في النهاية على الكتل الفعالة للناقلات. هذه الخصائص ليست متأصلة في المادة ، ولكنها تعتمد على عوامل خارجية أيضًا ، وأبرزها الإجهاد والحصر الكمي
بالنظر إلى الطبيعة البلورية لمعظم مواد أشباه الموصلات ، سيكون هناك إجهاد كلما نمت مادتان لهما ثوابت شبكية بلورية مختلفة فوق بعضهما البعض  حتى أولئك الذين لديهم نفس ثابت الشبكة قد يتعرضون للضغط بسبب تأثيرات أخرى مثل وجود الشوائب أو إذا تم استخدامها في درجات حرارة مختلفة لها معاملات تمدد حراري متباينة. يحدث الحبس الكمي ، بدوره ، عندما يتم تقليل حجم مادة أشباه الموصلات في بُعد واحد أو أكثر إلى بضعة نانومترات. في هذه الحالة ، تصبح مستويات الطاقة المتاحة للناقلات كمية في اتجاه الحبس ، مما يؤدي أيضًا إلى تغيير كثافة الحالات. كلا الشرطين يحدثان في وقت واحد عند التعامل مع الآبار الكمية المتوازنة الانفعال$  (QW) $ \\
تم استخدام الآبار الكمومية لضبط خصائص امتصاص الخلايا الشمسية عالية الكفاءة على مدار العقدين الماضيين. أدت الحاجة إلى الأدوات مناسبة لدراستها في سياق  الخلايا الكهروضوئية إلى تطوير نماذج المحاكاة   والتي كانت من  $ Solcore  $
يتضمن نهج $ Solcore  $لتقييم خصائص $ QWs  $أولاً حساب تأثير الإجهاد  باستخدام 8 نطاقات $ Pikus-Bir~ Hamiltonian ، $ و معالجة كل مادة في الهيكل كمادة سائبة، ثم استخدام النطاقات المزاحة والكتل الفعالة لحل معادلة شودينجر 1D ،بعد المحاذاة الصحيحة بين الطبقات. وأخيرًا يُحسب معامل الامتصاص بناءً على كثافة الحالات ثنائية الأبعاد ، بما في ذلك تأثير الإكسيتونات
\subsection{  أداة  (8 نطاقاتKp ) Bulk 8-band kp :  }
يتضمن $ Solcore  $ ملف $ band ~Pikus–Bir~ Hamiltonian~  8  $ لحساب بنية النطاق للمواد السائبة تحت إجهاد ثنائي المحور ، مع الأخذ في الاعتبارالإقتران بين  التوصيل ، الثقب الثقيل $ heavy~ hole   $، ثقب  الخفيف $ light~ hole  $ و $ split-off~ bands  $ ، الدوال الخاصة والحالات الخاصة هي حلول للمعادلة التالية حيث  
$ H_i $
هي :
$ Pikus–Bir ~Hamiltonian   $
\begin{figure}[h!]
	\centering
	\includegraphics[width=0.9\linewidth]{Fig/Fig_II/-6014949394668959668_120}
	\caption{}
	\label{fig:-6014949394668959668120}
\end{figure}
\FloatBarrier
لحل هذا النظام نستخدم وظيفة linalig.eig الخاصة بال Numpy التي توفر لنا القيم والدوال الذاتية
\subsection{  معادلة شرودينجر في  بعد واحد :  }
بمجرد معرفة حواف النطاق والكتل الفعالة لكل مادة من المواد التي تشكل بنية البئر الكمومي يمكن حساب الخصائص الكمومية عن طريق حل معادلة شرودينجر ذات البعد الاول .يتم إنشاء مصفوفة ثلاثية الأضلاع بكتابة معادلة شرودينجر ذات الكتلة الفعالة المتغيرة عبر سلسلة من نقاط الشبكة تتوافق القيم الذاتية للمصفوفة مع مستويات الطاقة المسموح بها للنظام . النظام يحل عبر المعادلة التالية 
\begin{equation}
	H\psi_i= -s_i\psi_{i-1}+d_i\psi_i-s_{i+1}\psi_{i+1}=E\psi_i
\end{equation}
حيث :$ S_i  $و $ d_i  $تعتمد على تباعد الشبكة  $\Delta$ والكتل الفعالة $ m_i $ والجهد  $ V_i  $المعتمدة على الموضع .
\begin{equation}
	d_i=\frac{\bar{h}^2}{ 4\Delta^2 m_0}(\frac{1}{ m_{i-1}}+\frac{2}{ m_{i-1}}+\frac{1}{ m_{ i+1}}) +V_i
\end{equation}
\begin{equation}
	s_i=\frac{ \bar{h}^2}{4\Delta^2 m_0 }(\frac{1}{ m_{i-1}}+\frac{1}{ m_{ i-1}})
\end{equation}
يتم حل هذا النظام بإستخدام  الأدوات المتاحة في حزمة Scipy ولا سيما           
sparse.linalg.eigs
\section{   مصادر الضوء Sources light }
إن تحويل ضوء الشمس إلى كهرباء هو الهذف النهائي لأي خلية شمسية ،  وبالتالي فمن الضروري أن يكون هناك طريقة مناسبة لإنشاء ومعالجة وتعديل خصائص طيف الضوء. \\ 

تم تصميم وحدة مصدر الضوء$ Source\_light $ للتعامل بسهولة مع مصادر الضوء المختلفة ،لديه دعم مباشر ل: 
\begin{itemize}
	\item 
	انبعاث غاوسي ، نموذجي لليزر وانبعاث الضوء للثنائيات 
	\item 
	إشعاع الجسم الأسود المميز لمصابيح الهالوجين التي تحددها درجة الحرارة ، ولكنها تستخدم أيضًا في كثير من الأحيان لمحاكاة طيف الشمس ، قريب جدًا من مصدر الجسم  الأسود عند 5800 كلفن
	\item 
	الأطياف الشمسية القياسية: الطيف خارج الأرض $ AM0 $ والآخران الأرضيان ، $ AM1.5D $ و $ AM1.5G $ مثل المحددة بواسطة معيار $ ASTM G173-03 (2008) $ 
	\item 
	ماذج الإشعاع باستخدام الموقع والوقت و معلمات الغلاف الجوي لحساب الطيف الشمسي الاصطناعي. يشتمل $ Solcore  $على نموذجين: $ SPECTRAL2  $، تم تنفيذه بالكامل في $ Python  $، وواجهة لثنائيات $ SMARTS  $(التي يجب تثبيتها بشكل منفصل)
	\item 
	الإشعاعات المعرفة من قبل المستخدم ، المقدمة خارجيًا من قاعدة بيانات أو أي مصدر آخر ، مما يسمح بأقصى قدر من المرونة.
\end{itemize}
يجب الأخد في عين الإعتبار نوع المصدر الذي يجب إنشاؤه  ،في حالة نماذج الإشعاع ، والتي غالبًا ما تحتوي على عدد كبير من المدخلات ، يحدد Solcore مجموعة من القيم الافتراضية ، لذلك لا يلزم توفير سوى العناصر المختلفة. توضح الصورة قائمة الكود  لإنشاء العديد من مصادر الضوء باستخدام الحد الأدنى من المدخلات المطلوبة في كل حالة.  
\begin{figure}[h!]
	\centering
	\includegraphics[width=0.7\linewidth, height=0.5\textheight]{"Fig/Fig_II/Exemple of the use of the LightSource class"}
	\caption{مثال لإستخدام صنف lightSource}
	\label{fig:exemple-of-the-use-of-the-lightsource-class}
\end{figure}
\FloatBarrier
\section{    المحاليل الضوئية optical solver }
الغرض من هذه المحاليل هو الحصول على إنعكاس الضوء الوارد وإمتصاصه ونقله في الخلية الشمسية كدالة للطول الموجي للضوء والموقع داخل الهيكل 
يتضمن Solcore ثلاثة نمادج لمعالجة هذه المشكلة : 
\subsubsection{قانون بير لامبيرت $ Beer–Lambert ~law (BL) $}

هو أبسط نموذج لحساب الامتصاص في هيكل متعدد الطبقات. يتجاهل كل الإ نعكاس في واجهات التي يمكن توفير انعكاس السطح الأمامي خارجيًا ، ويكون صفرًا  \\
يستخدم قانون BL على نطاق واسع في الخلايا الكهروضوئية ولكن في الواقع ، لا يمكن تطبيقه إلا عندما يمكن تجاهل التباين في معامل الانكسار بين الطبقات وعندما يكون هناك امتصاص قوي \\
الإمتصاص لكل وحدة طول كدالة لطول الموجة  $\lambda$ و الموضع z في الطبقة n يعطى بواسطة :\\
\begin{equation}
	A_n(\lambda,z)=\alpha_n(\lambda)exp(-\sum_{i=1}^{ n-1}\alpha_i(\lambda)w_i-\alpha_ n ( \lambda )(z-z_n))
\end{equation}
حيث : $\alpha_n$ هي معامل إمتصاص الطبقة n  ،سمكها  $ w_n $ ، موضع بداية الطبقة $  z_n $
\subsubsection{ طريقة مصفوفة النقل transfer matrix method (TMM)}

من أجل تقييم  السلوك الضوئي لخلية شمسية من المهم مراعاة تفاعل الإشعاع الكهرومغناطيسي المتولد مع تعاقب كل من الطبقات المستوية  الممتصة و الغير الماصة ،$ Solcore  $ يقوم بتقييم تفاعل الإشعاع الكهرومغناطيسي الساقط من خلال هيكل متعدد الطبقات بإستخدام TMM  \\
نستخدم نمودج TMM  في Solcore عن طريق وحدة tmm التي طورها Byrnes  
\subsubsection{   تحليل الموجة المزدوجة الصارمrigorous coupled-wave analysis (RCWA) }

يشتمل $ Solcore  $على واجهة $ S4~ solver  $  -التي يجب تتبيثها بشكل منفصل - والتي تم تطويرها في جامعة Stanford University ، من أجل تصميم الخلايا الشمسية النانوية . S4 هو تطبيق لRCWA والتي يشار إليها أيضا في بعض الأحيان بإسم طريقة فورييه النموذجية FMM  والتي تحل معادلات ماكسويل الخطية  في الهياكل التي تحتوي على دورية ثنائية الأبعاد 

يمكن حساب الضوء المنعكس والممتص والمرسل خارجيا تم توفيره كمدخل إلى Solcore  للحصول على الخواص الكهربائية لهيكل الخلايا الشمسية 
\section{ خلايا شمسية أحادية الوصلة} 
يشتمل  Solcore على أربعة أدوات حل لحساب الخصائص الكهربائية لجهاز أحادي الوصلة من أجل زيادة الدقة وهي :
\begin{itemize}
	\item  التوازن المفصل detailed balance
	\item 2-معادلة الصمام الثنائي 
	\item تقريب النضوب depletion approximation
	\item بواسون - الإنجراف - الإنتشار  Poisson–drift–diffusion
\end{itemize}
\subsection{ التوازن المفصل  detailed balance (BD) :}
يحسب هذا الحل الخواص الكهربائية للوصلة عن طريق موازنة العمليات الأولية التي تحدث في الخلية الشمسية وتوليد الناقل وإعادة التركيب الإشعاعي بإستخدام الشكلية التي وصفها أراجو ومارتي 1996. يتم إستخدام هذه الطريقة على نطاق واسع من قبل المجتمع الكهروضوئي لحساب كفاءة التحويل المحدودة لمواد الخلايا الشمسية المختلفة . أبسط صيغة لل$ DB $ تحتاج فقط إلى طاقة حافة الإمتصاص وقيمة إمتصاص  أعلى من تلك الحافة \\
يتم حساب إعادة التركيب الإشعاعي أو تيار التوليد الحراري من الخلية الشمسية بإتباع الشكلية التي وصفها نيلسون وأخرون مع الأخد في الإعتبار جميع المسارات الممكنة للضوء الذي تمتصه الخلية والعلاقة بين الإنبعات والإمتصاص يتم الحصول على إجمالي التيار الإشعاعي بإستخدام معادلة بلانك المعممة :
\\
======================
\\
حيث : A(E,0,s)  هو احتمال وجود فوتون من الطاقة E سوف ينبعث (أو يمتص)  من النقطة الموجودة على السطح بزاوية داخلية $\theta$ ، 
بالنسبة ل$ A~(front)  $ و  $ A~ (back) $ هما الإحتمال المشترك من الفوتون المراد إنبعاثه -إمتصاصه - من الجزء الأمامي والخلفي للخلية  على التوالي
\subsubsection{2-معادلة الصمام الثنائي   نمودج ثنائي الأبعاد  2D :} 
هذه هي أبسط طريقة لمحاكاة سلوك خلية شمسية باستخدام مكونات كهربائية لنمذجة آليات النقل وإعادة التركيب المختلفة . يتم تطبيق النموذج ثنائي الأبعاد على نطاق واسع عند نمذجة الخلايا الشمسية   ، عندما تكون المعرفة التفصيلية لهيكل الخلية الشمسية (الطبقات ، ومعامل الامتصاص ، وما إلى ذلك) غير معروفة أو مطلوبة.غالبا ما يتم إستخدامه لملائمة منحنيات IV التجريبية والعثور على معلومات عامة وتقريبية عن جودة الخلايا الشمسية دون الدخول في العمليات الأساسية  
المعادلة هي :
\begin{equation}
	J=J_{sc}-J_{01}( e^\frac{ q(V-R_s J)}{ n_1 k_b T}-1)-J_{02}( e^\frac{ q(V-R_s J)}{ n_2 k_b T}-1) -\frac{ V-R_s J}{ R_sh}
\end{equation}
حيث : $ J_{01} $ و$ J_{02} $ تيارات التشبع العكسية ، n1 و n2 عوامل المثالية ، $ R_{sh }$ مقاومة التحويل ،$ R_s $ 
مقاومة السلسلة
\subsection{ بواسون - الإنجراف - الإنتشار     Poisson–drift–diffusion   (PDD)}
تحل هذه الطريقة معادلة بواسون للجهد الكهروستاتيكي مع معادلات النقل للإلكترونات والثقوب والظروف الحدودية المناسبة في الحالة المستقرة ،إنها طريقة قياسية لحساب الخصائص الكهربائية لمعظم أجهزة أشباه الموصلات ، بما في ذلك الخلايا الشمسية أو الترانستورات أو الثنائيات الباعثة للضوء وهي الطريقة الوحيدة المضمنة في معظم حزم البرامج لمحاكاة أجهزة أشباه الموصلات مثل :$ PC-1D  $و $ Nextnano $ و$ AFORS-HET $.
توفر حزمة PDD جميع الأدوات اللازمة لبناء هيكل خلية شمسية وحساب خصائصها عن طريق حل معادلة Poisson ومعادلة انتشار  والإنجراف في نفس الوقت. عادةً ، لن تحتاج إلى الوصول إلى هذه الوظائف مباشرةً ، ولكن يتم استدعاؤها داخليًا بواسطة Solcore عند استخدام أساليب المستوى الأعلى في محلل الخلايا الشمسية  solar cell solver
\\
لإستخدام  حزمة PDD يكفي تضمين السطر التالي في التعليمات البرمجية الخاصة بك : 
\begin{flushleft}
	$ import ~solcore.poisson\_drift\_diffusion ~as~ PDD $
\end{flushleft}
مع هذا ستكون جميع وظائف الحزمة متاحة للمستخدم تنتشر الوظائف والحسابات الفعلية في عدة وحدات :
\subsubsection{ وحدة $ Drift ~diffusion ~utilities $ :}
الملف : solcore/PDD/DriftDiffusionUtilities.py
\\
يحتوي على واجهة python التي تقوم بتفريغ جميع معلومات بنية الجهاز في متغيرات Fortan والتي تنفذ الحساب المختار والحصول على البايانات من متغيرات Forten  في النهاية تحتوي هذه الوحدة على الطرق التالية :
\begin{flushleft}
	$ equilibrium\_pdd $ \\
	$ short \_circuit\_pdd $\\
	$ iv\_pdd $ \\
	$ qe\_pdd $  \\
\end{flushleft}
\subsubsection{وحدة $ Device~ structure $ :}
الملف : 
solcore/PDD/DeviceStructure.py
\\
يحتوي على العديد من الوظائف الضرورية لبناء هيكل يمكن قراءته بواسطة محلل PDD أهم هذه الوظائف هو create Device structure الذي يقوم بمسح طبقات الوصلات ويستخلص من قاعدة بيانات المواد جميع المعلمات المطلوبة بواسطة محلل PDD يحتوي على الخصائص الإفتراضية التي يتم إستخدامها إذا لم يتم العثور عليها لمادة معينة بشكل عام تتوافق هذه الخصائص الإفتراضية مع GaAs عند 293 كلفن 
\subsubsection{وحدة $ QW ~ unit ~creator $ :}
الملف : 
solcore/PDD/QWunit.py
\\
يحتوي على أدوات مساعدة تحول تسلسل الطبقات إلى بنية يمكن إستخدامها بدورها لحل معادلة شرودينجر ونمودج K.P ، يقوم أيضا بإعداد خصائص الهيكل ( النطاقات ، الكثافة الفعالة للحالات وما إلى ذلك ) ،من أجل الحصول على مجموعة ذات مغزى من الخصائص لمحلل DD ، باستخدام GetEffectiveQW
\subsubsection{وحدة  $  Drift ~Diffusion ~Fortran ~solver  $ :}
الملف : 
solcore/PDD/DDmodel-current.f95
\\
هذه هي النسخة الحالية من كود فورتران Fortran  والتي بمجرد تجميعها تقوم بتنفيذ جميع العمليات الحسابية العددية الثقيلة. عادة لن تحتاج إلى الاهتمام بهذا الملف إلا إذا كنت تنوي تعديل الحل العددي نفسه.
\subsubsection{تقريب النضوب $ depletion ~approximation $ :}
يوفر  تقريب  النضوب حلاً تحليليًا - أو شبه تحليلي - لمعادلات Poisson-drift-diffusion الموصوفة في القسم السابق المطبق على الخلايا الشمسية   PN البسيطة. الأهم من ذلك ، أنها تتطلب معلمات إدخال أقل من أداة حل PDD ويمكن ربطها بسهولة بكميات قابلة للقياس ، مثل التنقل -الحركية - أو أطوال الانتشار.يعتمد نموذج DA على افتراض أنه حول التقاطع بين منطقتي P و N ، لا توجد ناقلات حرة ، وبالتالي فإن كل المجال الكهربائي يرجع إلى الشوائب  المؤينة الثابتة.يصل هذا "النضوب" للحاملات الحرة إلى عمق معين باتجاه الجانبين N و P  خارج هذه المنطقة ، في ظل هذه الظروف ، تنفصل معادلة بواسون عن معادلات الانجراف والانتشار ويمكن حلها تحليليًا لكل منطقة.\\
نتيجة أخرى لتقريب النضوب هي أن طاقات مستوى شبه فيرمي ثابتة في جميع أنحاء المناطق المحايدة المقابلة وثابتة أيضًا في منطقة النضوب ،تبسط معادلات الانجراف-الانتشار ويمكن العثور على تعبير تحليلي لاعتماد تيارات إعادة التركيب والتوليد على الجهد المطبق
يتضمن تنفيذ Solcore لتقريب النضوب  هو السماح بإدراج  منطقة جوهرية  بين منطقتي P و N لتشكيل وصلة  PIN بالنسبة لظروف الإندار المنخفض - الإضاءة المنخفضة أو التحيز المنخفض - يمكن معالجة هذا الموقف مع الأخد في الإعتبار ببساطة أن منطقة النضوب تتسع الأن بسمك المنطقة الجوهرية لا يسمح بمستوى منخفض من المنشطات لهذه المنطقة بالإضافة إلى أنه في Solcore ، نقوم بدمج التعبيرات الخاصة بمعادلات الانجراف-الانتشار تحت تقريب التضوب عدديًا للسماح بملف تعريف توليد   محسوبًا بأي من المحاليل  الضوئية إلا أنه لن يكون حلا متكاملا ذاتيا لمعادلات Poisson–drift–diffusion كما تم تحقيقه بواسطة محلل    
PDD
\section{ الخلايا الشمسية متعددة الوصلات}
يمكن أن تشتمل الخلية الشمسية الكهروضوئية الكاملة على وصلة واحدة أو أكثر ، وملامسات معدنية ، وطبقات بصرية (بما في ذلك الطلاءات المضادة للانعكاس والتركيبات النانوية الضوئية) . قد تتراوح الوصلات  ، بدورها ، من تماثلات PN البسيطة إلى الوصلات غير المتجانسة المعقدة ، بما في ذلك هياكل الآبار متعددة الكم. المحاليل -أدوات الحل - الموصوفة حتى الآن تحسب فقط خصائص أجهزة الوصلة الواحدة. لدمجها في جهاز متعدد الوصلات ، من الضروري مراعاة أن الوصلات الفردية متصلة كهربائيًا في سلسلة والاقتران المحتمل للضوء المنبعث من الوصلات ذات فجوة الحزمة الأوسع مع تلك ذات فجوة النطاق الأصغر
\subsubsection{ عدم وجود إقتران إشعاعي :}
نعتبر أولاً حالة عدم وجود اقتران إشعاعي بين الوصلات. هذا تقدير تقريبي جيد للمواد التي لا تشع بكفاءة أو المواد المشعة التي تعمل بتركيز منخفض ، عندما يكون جزء إعادة التركيب الإشعاعي مقارنة بإعادة التركيب غير الإشعاعي منخفضًا. في هذه الحالة ، يمكن حساب المنحنى IV لكل  وصلة  بشكل مستقل عن بعضها البعض ويكون التيار المتدفق عبر هيكل MJ مقيدًا بالوصلة  مع أدنى تيار عند أي جهد معين. يتم الآن إضافة مقاومات التسلسل المحددة لكل وصلة  معًا وتضمينها كمصطلح واحد. يتم أخيرًا حساب جهد التشغيل لكل من الوصلات وإضافته معًا للحصول على جهد جهاز MJ
\subsubsection{مع إقتران إشعاعي :}
يحدث الاقتران الإشعاعي عندما يتم امتصاص الضوء الناتج عن وصلة  ذو فجوة نطاق عالية بسبب إعادة التركيب الإشعاعي بواسطة وصلة ذات فجوة نطاق منخفضة ، مما يساهم في تياره الضوئي، تم التعرف عليه في العديد من المواد عالية الإشعاع مثل الخلايا الشمسية للبئر الكمي والخلايا الشمسية $ III-V MJ   $\\
تستند شكليات الاقتران الإشعاعي المضمنة في Solcore إلى أعمال  chan و al  ونيلسون وآخرون. (1997).  \\
يتم تنفيذه بواسطة نمودج DB و 2D عندما يتم تعريفه من حيث الكفاءة الإشعاعية والمعلمات المحسوبة من نموذج قاعدة البايانات  
\subsection{  القيود في تعريفات الوصلات :  }
توفر طرق متعددة لنمدجة الوصلات الكثير من الحرية والمرونة ، ولكنها تفرض أيضا بعض القيود في كيفية دمجها من أجل إنشاء خلية شمسية   MJ  منها :
\begin{itemize}
	\item  عندما لايكون هناك إقترا إشعاعي ونحن مهتمون فقط بخصائص IV المظلمة  يمكن دمج جميع نماذج الوصلات مع بعضها البعض حيث يتم تحديد الوصلة العلوية بإستخدام نمودج   DB ويتم تحديد الوصلة الوسطى  بنمودج 2D ويتم تحديد الوصلة السفلى بإستخدام نمودج PDD
	\item  الأمر نفسه ينطبق على محاكاة   IV الضوء  والكفاءة الكمومية  طالما أن النموذج البصري المختار هو قانون BL في هذه الحالة  يجب أن يتضمن أي وصلة تم تحديدها بإستخدام النمودج 2D قيمة الإمتصاصية
	\item يتم تدعيم النماذج البصرية TMM و RCWA بواسطة نماذج الوصلة PDD و DA
	\item في  حالة وجود اقتران إشعاعي ،  فإن نماذج  الوصلات الوحيدة التي يمكن استخدامها هي DB و 2D ، طالما أن  ملف  الأخير يتضمن  قيمة امتصاصية
\end{itemize}\begin{itemize}
	\item  عندما لايكون هناك إقترا إشعاعي ونحن مهتمون فقط بخصائص IV المظلمة  يمكن دمج جميع نماذج الوصلات مع بعضها البعض حيث يتم تحديد الوصلة العلوية بإستخدام نمودج   DB ويتم تحديد الوصلة الوسطى  بنمودج 2D ويتم تحديد الوصلة السفلى بإستخدام نمودج PDD 
	\item  الأمر نفسه ينطبق على محاكاة   IV الضوء  والكفاءة الكمومية  طالما أن النموذج البصري المختار هو قانون BL في هذه الحالة  يجب أن يتضمن أي وصلة تم تحديدها بإستخدام النمودج 2D قيمة الإمتصاصية
	\item يتم تدعيم النماذج البصرية TMM و RCWA بواسطة نماذج الوصلة PDD و DA
	\item في  حالة وجود اقتران إشعاعي ،  فإن نماذج  الوصلات الوحيدة التي يمكن استخدامها هي DB و 2D ، طالما أن  ملف  الأخير يتضمن  قيمة امتصاصية  
\end{itemize}
\section{ خاتمة }
في هذا الفصل تم التطرق إلى قدرات Solcore ونماذج التي يعتمد عليه في حل - دراسة - مواد أشباه الموصلات التقليدية والجديدة والخلايا الشمسية من عناصر قوته :
\begin{itemize}
	\item 
	
	المرونة توفر مجموعة متنوعة من الأدوات بدلا من حل واحد لدراسة مواد وأجهزة أشباه الموصلات  
	\item  
	إمكانية الوصول ليست مفتوحة المصدر فحسب ، بل إنها مصممة أيضا لتكون سهلة التعلم والإستخدام ، حيث تعمل كأداة تعليمية بقدر ما تعمل كأداة بحث 
\end{itemize}
في فصلنا التالث والأخير سيتم  فيه إستغلال هذا البرنامج الجميل لدراسة خصائص خلية شمسية مصنعة من ErGaN و  ErN


