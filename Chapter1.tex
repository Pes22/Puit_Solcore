\chapter{مقدمة حول الخلايا الشمسية ذات الآبار الكمومية}

\label{Chapter1}


\section {تمهيد }  
بما أن الطاقة الشمسية هي أهم مصادر الطاقة المتجددة خلال القرن القادم فإن جهود كثير من الدول تتوجه بمختلف صورها وترصد لها المبالغ اللازمة لتطوير المنتجات والبحوت الخاصة بإستغلال الطاقة الشمسية كإحدى أهم مصادر الطاقة البديلة للنفط والغاز ،وقد أعطى النصيب الأوفر في البحوث والتطبيقات لمجال تحويل الطاقة الشمسية إلى كهرباء وهومايعرف بالخلايا الشمسية  $ photovoltics   $             
في هذا الفصل سوف نقدم عرضا للعديد من النقاط الهامة والمحورية التي تتعلق بالخلايا الشمسية مثل الإشعاع الشمسي  وطيفه مع الإشارة إلى المبدأ الأساسي لعملها التأثير الكهروضوئي وذكر بعض المفاهيم للمكون الأساسي لها وهي أشباه النواقل كتعريفها وأنواعهابالإضافة إلى ذكر وصف عام للخلايا الفوتوفولطية وأنواعها وخصائصهاثم الإنتقال إلى تعريف نهج جديد في الخلايا الشمسية وهي الخلايا ذات الأبار الكمومية \cite{a2} .
\section {الإشعاع الشمسي  }
إن حوالي 40 ٪ من الإشعاع الشمسي الواصل إلى الأرض ينعكس إلى الفضاء بسب الغيوم والغلاف الجوي وبعض السطوح كالماء والثلج والرمل، والجزء الأخر من الإشعاع الشمسي في أثناء مروره عبر الغلاف الجوي يتبعثر في كل الإتجاهات، يسقط قسم من هذا الإشعاع المبعثر على سطح الأرض وتقوم الغيوم والغبار ببعثرة قسم منه ، والقسم المتبقي يقوم بامتصاصه بخار الماء وثاني أكسيد الكربون والأوزون الموجود في الغلاف الجوي . وتبلغ قيمة معدل الاشعاع الشمسي الساقط على المحيط الخارجي للأرض (1357W/m²) وهو ما يعرف بالثابت الشمسي .
تتكون مجموعة الاشعاعات التي ترتطم بسطح الخلية الكهروشمسية وبمساحة معينة على سطح الأرض كما هو موضح في الشكل من ثلاثة اجزاء اساسية وهي حزمة الاشعاع المباشر ، حزمة الاشعاع المبعثر ، حزمة الاشعاع المعكوس.  [1]

\begin{figure}[h!]
	\centering
	\includegraphics[width=0.8\linewidth, height=0.5\textheight]{Fig/Fig_I/sun}
	\caption{الاجزاء الاساسية للطيف الشمسي}
	\label{fig:sun}
\end{figure}
\FloatBarrier

\subsection{الطيف الإشعاعي}
يتكون الاشعاع الشمسي من موجات كهرومغناطيسية تقسم الى قطاعات حسب اطوالها الموجية ، الا ان ما يقارب 98٪ منه يتكون من الأشعة وهي :
\begin{itemize} 
	\item  الأشعة فوق البنفسجية 8\% ويغطي هذا النوع من الأشعة جزء من الطيف الشمسي الذي يحتوي على الأشعة ذات الموجات القصيرة حتى طول 0.4 ميكرون 
	\item الأشعة المرئية (47٪) ويتراوح طول موجتها بين 0.4 و0.75 ميكرون 
	\item الأشعة تحت الحمراء (43٪) حيث يزيد طول موجتها عن 0.75         ميكروون [2]
\end{itemize}
\begin{figure}[h!]
	\centering
	\includegraphics[width=0.8\linewidth, height=0.5\textheight]{Fig/Fig_I/spector}
	\caption{الطيف الاشعاعي}
	\label{fig:spector}
\end{figure}

\subsection{ تفاعل ضوء الشمس مع المادة }
نظريات الجسيمات الضوئية تسمح لنا بتفسير السلوك الفيزيائي المتبادل للضوء مع المادة .
تختلف أساليب انتقال الطاقة من الاشعاعات إلى المادة اختلافا جوهريا عن تلك الأساليب التي تنتقل بها الجسيمات المشحونة إلى المادة . فعندما يسقط فوتون على المادة فإنه يمكن أن يفقد طاقته ويمنحها للمادة عن طريق إحدى العمليات الثلاث الرئيسة التالية : التأثير الكهروضوئي ، تاثير كومبتون ، وانتاج الأزواج 
\begin{itemize} 
	\item التأثير الكهروضوئي : نتيجة للتصادم المباشر بين الفوتون الساقط و احد الالكترونات المرتبطة بالذرة تنتقل طاقة الفوتون بإكملها إلى ذلك الالكترون الذي ينطلق تاركا ذرته . ويسمى هذا الإلكترون المنطلق بالإلكترون الكهروضوئي
	[3] 
	\begin {equation} 
	E_{e} = E_{\gamma} - B = h\nu -B
	\end {equation}
	\end {itemize}
	\subsection{استخدامات الطاقة الشمسية}	
	استفاد الإنسان منذ القدم من طاقة الإشعاع الشمسي مباشرة في تطبيقات عديدة كتجفيف المحاصيل الزراعية وتدفئة المنازل ، كما استخدمها في مجالات أخرى وردت في كتب العلوم التاريخية ، فقد أحرق أرخميدس الأسطول الحربي الروماني في حرب عام (202) قبل الميلاد عن طريق تركيز الإشعاع الشمسي على سفن الأعداء بواسطة المئات من الدروع المعدنية ، وفي العصر البابلي كانت نساء الكهنة يستعملن آنيات ذهبية مصقولة كالمرايا لتركيز الاشعاع الشمسي للحصول على النار 
	حاول الإنسان منذ فترة بعيدة الاستفادة من الطاقة الشمسية واستغلالها ولكن بقدر قليل ومحدود ، ومع التطور الكبير في التقنية والتقدم العلمي الذي وصل إليه الإنسان فتحت آفاق علمية جديدة في ميدان استغلال الطاقة الشمسية
	[1]
	\begin{enumerate}
		\item الإستخدامات الحرارية للطاقة الشمسية  
		:
		هي عملية حصاد واستغلال الطاقة الشمسية لإنتاج طاقة حرارية وتستخدم المركزات أو المجمعات الشمسية الحرارية ( Solar thenmal collector ) لهذا الغرض .
		ومن أهم الاستخدامات انتاج الملح من مياه البحار بالسماح لجزيئات الماء بالتبخر وترسيب الأملاح ، انتاج مياه حارة للاستخدامات المنزلية والتجارية عن طريق المساحات الشمسية بالإضافة إلى التجفيف ( تجفيف الخشب لإنتاج الوقود والفحم ) والطبخ والتقطير
		.
		[4]
		\item استخدامات الطاقة الشمسية لتوليد الكهرباء 
		:
		وقد استخدمت الطاقة الشمسية لتوليد الكهرباء في تطبيقات عديدة منها محطات توليد الكهرباء وتحلية المياه، وتشغيل إشارات المرور وإنارة الشوارع، وتشغيل بعض الأجهزة الكهربائية مثل الساعات . والآلات الحاسبة، وتشغيل الأقمار الإصطناعية والمركبات والمحطات الفضائية، ومؤخرا رأينا على التلفاز سيارة تسير بالطاقة الشمسية تصل سرعتها إلى 60 ميل (96 كم) في الساعة.
		[5]
	\end{enumerate}
	\section {أشباه النواقل }
	
	
	بين المواد العازلة و المواد الموصلة يوجد مادة في غاية الأهمية هي أشباه الموصلات (6) ، بها نوعين مادة شبه موصلة في الحالة النقية (الجوهرية) وهي ليست موصل جيد ولا هي عازل جيد للكهرباء و النوع الثاني شـبـه الموصل غير النقي (الخارجي أو المشـوب) علاوة على ما سبق توجـد المادة شبه الموصلة إما فردية العنصر ( مثل  الأنتيمون $ ( Sb)، $ الزرنيخ $ ( As )، $ والسيليكون $ (Si)، $ والجرمانيوم  $ (G e) $)،أو تكون مركبة من أكثر من عنصر وتسمى أشباه الموصلات المركبة ثنائية العنصر فورتسیت شبه موصل وهي شائعة الاستخدام أيضا ، منها زرنيخيد الجاليوم، نيتريد الجاليوم، وأكثر العناصر الفردية شيوعا عنصري الجرمانيوم والسيليكون وهما عبارة عن عناصر رباعية التكافؤ (يحتوي على غلاف الذرة الخارجي 4 إلكترونات )‬،ترتبط ذراتها ببعضها البعض بروابط تساهمية عن طريق إلكترونات المدار الخارجي وتكون عازلة تماما عند درجة الصفر المطلق،بحيث لاتوجد أية إلكترونات حرة في منطقة التوصيل لنقل الكهرباء،‫لكن من الممكن أن تتحطم بعض الروابط التساهمية وذلك بتطبيق طاقة حرارية أو ضوئية متزايدة القمة عن نوع هذه البلورة،وهذا التحطيم يؤدي إلى إهتزاز الذرات وتتحرر بعض الإلكترونات من مداراتها الخارجية فتولد فراغات في البلورة وتسمى هذه الفراغات بالفجوات أو الثقوب  وتمتاز أشباه الموصلات بكفاءتها في مجال الطاقة، وبانخفاض أسعارها ، لذلك فهي تستخدم على نطاق واسع في مجال صناعة الأجهزة الإلكترونية، بما في ذلك الديودات  $ (Diodes)، $ والترانزستورات $ (Transistor )$، والدوائر المتكاملة 
	($ Integrated circuits)   $
	[2,6,7,8]
	
	\subsection{ بلورة نصف ناقلة ذاتية}
	يكـون شـبـه الموصـل نقيـا إذا كانت الغالبية العظمى مـن حـاملات الشحنة الحرة (الإلكترونات والفجوات) ناشئة من ذرات شبه الموصل ذاته ، مثلا لدينا في بلورة السليكون كل ذرة سيليكون تمتلك أربع إلكترونات على مدارها السطحي،حيث تتشارك كل ذرة مع جارتها بهذه الإلكترونات لتمتلك كل ذرة ثمانية إلكترونات في طبقتها السطحية،وبالتالي تصل لحالة الإستقرار ،ففي هذه الحالة الذرة لن تتخلى عن أي إلكترون من إلكتروناتها بسهولة،ويلزمنا طاقة لانتزاع هذه الإلكترونات،قد تكون هذه الطاقة ببساطة ارتفاع بسيط في درجة حرارة المادة عبر تسخينها (7،10)
	
	
	\subsection{  بلورة نصف ناقلة غير ذاتية (مطعمة)} 
	\begin{itemize}
		\item الاشابة (التطعيم) : 
		في العادة، يتم إدخال الشوائب المانحـة والمتقبلـة إلى أشباه الموصلات لزيادة تركيزات الإلكترونات أو الفجوات، الامر الذي يعدل الخواص الكهربية للمادة. إن مستويات الطاقة التي يتم إنشاؤها في فجوة الطاقة بواسطة مثل هذه الشوائب تقع على مقربة من قمة شريط التكافؤ أو قاع شريط التوصيل. عناصر أخرى تدخل مثل الذهب والحديد والنحاس والزنك مستوى طاقة واحد أو عدة مستويات في فجوة الطاقة لبلورة السيليكون. تقع هذه المستويات قريبة من وسط فجوة الطاقة وتسمى "مستويات عميقة" (عميقة لانها تكون في عمق فجوة الطاقة، أي بعيـدة عـن حـواف الفجـوة). عـادة يكـون لهذه الأخيرة تـأثير ضـار علـى أشـباه الموصلات، وهذا هو السبب في أن صناعة أشباه الموصلات تستخدم بلورات بدرجة عالية جدا من النقاء . [7]
		
	\end{itemize}
	إن أكثر خواص نصف الناقل فائدة تكمن في حقيقة أن ناقليته يمكن أن تتغير من دون إثارة،وذلك بحقن شوائب (تطعيم) في صميم البلورة،وتسمى عنذئذ أنصاف نواقل خارجية ، وتظهر هذه الناقلية الخارجية على أحد الشكلين:
	[11]
	\begin{enumerate}
		
		\item   بلورة شبه موصل غير نقي من النوع السالب 
		$ N type $ (أو عصبة التكافؤ) :
		عبارة عن بلورة شبه موصل نقي  ليكن جرمانيوم (Ge) أو سيليكون (Si) مخلوطة ببعض ذرات شائبة خماسية التكافؤ مثل الزرنيخ(تحتوي في مستوى الطاقة الأخير على خمس إلكترونات) وفي هذا النوع من البلورات تكون كل ذرة شائبة"خماسية التكافؤ"وليكن الزرنيخ مرتبط بأربع ذرات سيليكون عن طريق أربع روابط تشارك فيهم ذرة الزرنيخ بأربع إلكترونات ويتبقى الإلكترون الخامس لذرة الزرنيخ ضعيف الإرتباط بها إلكترون حر.كلما زادت عدد الذرات الشائبة يزداد عدد الإلكترونات الحرة وبالتالي تزداد قدرة البلورة على توصيل التيار الكهربائي.ويسمى هذا النوع من البلورات ببلورة سالبة لأن خاصية التوصيل الكهربائي بها ناتجة عن حركة الإلكترونات السالبة .ويسمى هذا النوع من ذرات الشوائب بالذرات المانحة .
		
		\item    بلورة شبه موصل غير نقي من النوع الموجب
		$ p type $:(أو عصبة التوصيل )  
		عبارة عن بلورة شبه موصل نقي  وليكن جرمانيوم (Ge) أو سيليكون (Si) مخلوطة ببعض ذرات شائبة ثلاثية التكافؤ مثل( الجاليوم والألميوم والأنديوم) تحتوي في مستوي الطاقة الأخير على ثلاث إلكترونات.وفي هذا النوع من البلورات تكون كل ذرة شائبة "ثلاثية التكافؤ"وليكن الجاليوم مرتبط بأربع ذرات سيليكون عن طريق أربع روابط تساهمية تشارك فيهم ذرة الجالييوم بثلاث إلكترونات ويتبقى في الرابطة مكان للإلكترون غير موجود يسمى فجوة تقوم هذه الفجوة  بجذب إلكترون من رابطة مجاورة وعندما ينتقل الإلكترون يملأ هذه الفجوة ويترك خلفه فجوة جديدة وهكذا.ونتيجة لحركة الإلكترون بين الروابط يملأ الفجوة يتسبب ذلك في وجود إلكترونات حرة مما يجعل البلورة توصل التيار الكهربائي 
		[12]
		
		\begin{figure}[h!]
			\centering
			\includegraphics[width=0.8\linewidth, height=0.5\textheight]{"Fig/Fig_I/ P_N"}
			\caption{شبه موصل موجب وشبه موصل سالب}
			\label{fig:-pn}
		\end{figure}
		\FloatBarrier	
		
	\end{enumerate} 
	\subsection{  الوصلة  PN}
	نحن نعلم أن شريحة $ p type $تمتلك أغلبية من الفجوات التي أنتجت بفعل عملية التطعيم التي ذكرناها سابقا،وأيضا تملك عددا قليلا من الإلكترونات الحرارية التي أنتجتها الطاقة الحرارية.وعلى هذا فإن شريحة Ntype تملك أغلبية من الإلكترونات، مع أقلية من الفجوات التي أنتجتها الطاقة الحرارية.
	بسبب التقنيات الصناعية المختلفة استطاع المصنعون إنتاج بلورة منفردة بنوع واحد من مادة شبه الموصل في جانب واحد، بينما الجانب الأخرمن البلورة يحتوي على النوع الأخر من مادة شبه الموصل،في هذه اللحظة نرى أن الوصلة قد تشكلت،الفجوات مازالت متبقية في النوع الموجب لشريحة شبه الموصل،والإلكترونات الحرة هي الأخرى مازالت متبقية في النوع السالب لشريحة الموصل، هذه الشحنات لها الحرية في حركة،وكنتيجة لذلك فإن هذه الحركة ستكون عشوائية في كل الإتجاهات.بعض من هذه الإلكترونات الحرة والفجوات سوف تتجه بإتجاه الوصلة junction و يحدث لها إرتباط $ recombine $،عند حدوث الإرتباط فإن الشحنات الحرة لن تظهر في منطقة الإرتباط،وعندها ستتحول منطقة الإرتباط الضيقة إلى ما يسمى بمنطقة النضوب $ Depletion ~region $.
	[13]
	\subsection{إنتقال الشحنات في أشباه الموصلات}
	توجد آليتان أساسيتان تتحكمان في مرور نواقل الشحنات (الإلكترونات والفجوات) عبر أشباه الموصلات
	\begin{itemize}
		
		\item    تيار الإزاحة 
		-الإنجراف - 
		:
		وهو حركة نواقل الشحنات نتيجة تأثير مجال كهربائي على طرفي مادة شبه موصلة،فالمجال الكهربائي ماهو إلا قوة تتسبب في تسارع الشحنات مولدا ما يعرف بالجهد الكهربائي، فأشباه الموصلات تسلك سلوك مشابه بعض الشيء لسلوك الموصلات.ومن أهم الفروق التي نلاحظها أن تولد التيار الكهربائي فيها قد يكون بسبب إنتقال الإلكترونات أو الفجوات أو الإثنين معا
		فقد تم إثبات العلاقة الطردية بين شدة المجال الكهربائي وسرعة النواقل عمليا: 
		\begin {equation} 
		V= \mu E
		\end {equation}
		
		$\mu$ حركية الشحنات 
		\\
		E   المجال الكهربائي 
		
		من خلال هذه العلاقة  نستنتج علاقة أخرى وهي العلاقة بين المجال الكهربائي وكثافة التيار في أشباه الموصلات:
		\begin{equation} 
			J= J_n + J_p = q E (n \mu_ n + p \mu_ p )
		\end{equation}
		حيث $ n $  كثافة الإلكترونات 
		$ P $  كثافة الفجوات   
		يصف الحد الأول تيار الإلكترونات والحد الثاني تيار الفجوات
		\\  
		من تلك المعادلة نستنتج أن تياري الإلكترونات والفجوات رغم إتجاههما المتضادين إلا أن شحنتيهما المتضادتين يولدان تيارا كهربائيا في نفس الإتجاه بالضبط.وقد جرت العادة أن يكون إتجاه التيار الكهربائي هو إتجاه سريان الشحنات الموجبة (الفجوات) أو عكس إتجاه سريان الشحنات السالبة(الإلكترونات).
		[14]
		\item تيار الإنتشار  
		:
		يعرف الإنتشار بأنه العملية التي تتحرك فيها حاملات الشحنة في شبه الموصل سواء كانت إلكترونات أو فجوات من المنطقة الأعلى تركيز إلى المنطقة الأقل تركيز يسمى التيار الناتج عن حركة حاملات الشحنة بتيار الإنتشار ، يحدث الإنتشار في غياب تطبيق مجال خارجي .
		
		[7]
		
		تيار الإنتشار للإكترونات 
		\begin {equation} 
		F_n = q D_n \frac{dn}{ dx}
		\end {equation}  
		تيار إنتشار الفجوات 
		\begin {equation}
		F_p =-q D_p \frac{dn}{ dx}
		\end {equation}
		حيث
		$ D_n  $   ثابت نشر الإلكترون
		\\
		$ q $  شحنة الالكترون 
		
		$ D_p $  ثابت نشر الفجوات
	\end{itemize}
	\subsection{توليد وإعادة تركيب الإلكترونات والثقوب recombination-generation}
	
	
	\subsubsection{ التوليد}
	يقوم الإلكترون (أو الثقب) مع طاقة حركية كافية بإخراج إلكترون مرتبط من حالته المقيدة (في نطاق التكافؤ) وترقيته إلى حالة في نطاق التوصيل ، وبالتالي إنشاء زوج إلكترون وثقب فالإلكترون يحرر نفسه من الرابطة التساهمية إذا توفر له مايكفي من الطاقة من خلال ذلك يقفز الإلكترون من نطاق التكافؤ إلى نطاق التوصيل ويصبح حر التحرك في البلورة في هذه العملية تتولد فجوة حرة أيضا وهو مايسمى توليد زوج إلكترون - فجوة .تعتمد هذه العملية  بشكل أساسي على نوع شبه الموصل إذا كان ذاتي أو خارجي  وعلى تركيزات التطعيم ودرجة الحرارة .
	\\
	بالنسبة للخلايا الشمسية ، فإن عملية توليد الإلكترونات والثقوب عن طريق امتصاص الفوتونات هي أهم عملية حيث يتم تحديد إحتمالية إمتصاص طاقة فوتون $  h\nu $  عن طريق معامل الإمتصاص $  \alpha ( h\nu)$ فهذا الاخير يتناسب مع كثافة الحالات المشغولة في نطاق التكافؤ الذي يمكن فيه إنشاء ثقب ، والحالات غير المشغولة في نطاق التوصيل الذي يمكن أن يكون فيه الإلكترون قد  ولد.
	\begin{figure}[h!]
		\centering
		\includegraphics[width=0.8\linewidth, height=0.5\textheight]{"Fig/Fig_I/ band to band generation"}
		\caption{التوليد من نوع نطاق نطاق}
		\label{fig:-band-to-band-generation}
	\end{figure}
	\FloatBarrier
	
	\subsubsection{إعادة التركيب أو الإتحاد}
	
	
	هي عملية معاكسة لعملية التوليد فالإلكترون  الموجودة في نطاق التوصيل يكون  حر الحركة  في البلورة وأثناء حركته يمكنه القفز إلى مقعد شاغر الفجوة  في نطاق التكافؤ  وبعمل ذلك  يطلق الإلكترون طاقة ففي هذه العملية يختفي كل من الإلكترون الحر والفجوة الحرة 
	من أهم عمليات إعادة التركيب :
	\begin {itemize}
	\item عملية إعادة التركيب الإشعاعي (الباعث للضوء) يتحد الإلكترون مع الفجوة باعتا فوتونا$ E=h\nu $  وهي مساوية لفرق الطاقة بين 
	النطاقين
	\begin{figure}[h!]
		\centering
		\includegraphics[width=0.7\linewidth, height=0.2\textheight]{"Fig/Fig_I/ radiative recombination"}
		\caption{اعادة التركيب الاشعاعي}
		\label{fig:-radiative-recombination}
	\end{figure}
	\FloatBarrier
	\item إعادة إرتباط أوجي : لا يشع إلكترون طاقة على شكل فوتون  وإنما قبل عملية الإرتباط يعطي طاقته الحركية  لإلكترون آخر ثم يفقد هذه الطاقة على شكل فونونات في تطادم مع الشبكة 
	
	\begin{figure}[h!]
		\centering
		\includegraphics[width=0.7\linewidth, height=0.4\textheight]{"Fig/Fig_I/ Auger recombination"}
		\caption{اعادة تركيب اوجير}
		\label{fig:-auger-recombination}
	\end{figure}
	\FloatBarrier
	
	\item  إعادة إرتباط شوكلي ريد - هول   إن أهم عمليات إعادة التركيب في أشباه الموصلات  هي تلك التي تنطوي على حالات خلل أو مصيدة في فجوة النطاق   فالناقل  الحر لإلكترون أو الفجوة يتم  التقاطه بواسطة المصيدة. يمكن بعد ذلك إطلاق الناقل عن طريق التنشيط الحراري (يتم تبادل الطاقة على شكل إهتزاز شبكي ) وتسمى تلك المصائد التي تلتقط  كلا النوعين من الناقلات  مراكز إعادة 
	التركيب
	[7،15،16]
	\end {itemize} 
	
	
	
	\subsection{ الإنتقالات المباشرة والغير مباشرة}
	\begin{itemize}
		\item  الإنتقالات المباشرة :
		يكون فيها قعر حزمة التوصيل وقمة حزمة التكافؤ في فضاء الموجة عند نفس النقطة K (متجه الموجة يمثل زخم حاملات الشحنة الثقوب والإلكترونات ) يسمى شبه الموصل الذي له هذا الإنتقال بشبه موصل ذو فجوة طاقة مباشرة . عند حوت عملية إعادة الإتحاد يكون الإلكترون له زخم K=0 والفجوة لها زخم K=0 وبما أن معظم الإلكترونات في نطاق التوصيل تكون عند أو بالقرب  K=0 فحدوث عملية إعادة الإتحاد تكون مرجحة جدا
		\item  الإنتقالات الغير مباشرة :
		هي التي يكون لا  فيها قاع نطاق التوصيل وقمة نطاق التكافؤ عند نفس قيمةK تسمى مثل هذه أشباه الموصلات "شبه موصل فجوة طاقة غير مباشر ، هذا النوع من الإنتقالات عندما يحدث إعادة الإتحاد في مثل هذه المواد فيعاد إتحاد إلكترون ذو زخم K=[Km,0] مع فجوة لها زخم K=0  يحدث بمساعدة الفونون من أجل الحفاظ على  الزخم
		[7]
	\end{itemize}
	\begin{figure}[h!]
		\centering
		\includegraphics[width=0.8\linewidth, height=0.5\textheight]{"Fig/Fig_I/ direct indirect"}
		\caption{فجوة النطاق المباشرة وغير المباشرة}
		\label{fig:-direct-indirect}
	\end{figure}
	\FloatBarrier
	\section{ الخلايا الشمسية}
	\subsection{ وصف عام للخلايا الشمسية-الخلايا الكهروضوئية  }
	‫الخلية الشمسية هي عنصر إلكتروني يقوم بإمتصاص الفوتونات الساقطة من أشعة الشمس، وتحويلها إلى تيار مستمر (DC)،  
	فالخلية الشمسية هي عبارة عن وصلة PN مصنوعة من أشباه الموصلات مثل السيليكون ،تولوريد الكادميوم والكاليوم أرسينايد ،حيث تكون فيها الطبقة n رقيقة ويتم تسليط الضوء عليها لكي تتولد فولتية بين طرفيها وتيار يسري في حمل خارجي ،وتعتمد شدته على مجموعة من العوامل:منها شدة الضوء الساقط على الخلية،وزاوية سقوطه،بالإضافة إلى المواد المستخدمة في تصنيع الخلية الشمسية.
	تعتمد تقنية الخلايا الضوئية على كفاءة الخلية وكلفة التصنيع.حيث يتم التركيز في الإنتاج والتطوير على تحسين الكفاءة والكلفة ﻷن الحل الأمثل يبنى على أساسهما، تحدد الكفاءة في الخلايا الكهروضوئية بقدرة المواد ضمن هذه الخلايا على إمتصاص طاقة الفوتون عبر مجال واسع،وعلى الفجوة الطاقية لهذه المواد [4،17،18].
	\begin{figure}[h!]
		\centering
		\includegraphics[width=0.8\linewidth, height=0.5\textheight]{"Fig/Fig_I/ photovoltics"}
		\caption{مبدأ عمل الخلية الشمسية}
		\label{fig:-photovoltics}
	\end{figure}
	\FloatBarrier
	\subsection{ أنواع الخلايا الشمسية  }
	توجد أنواع مختلفة من الخلايا الشمسية تختلف فيما بينها في طريقة التصنيع والمواد الداخلة في تصنيعها وقدرتها على إنتاج الطاقة الكهربائية وسعرها أيضا وأنواعها هي :
	
	
	\subsubsection{ الخلايا الشمسية من الجيل الأول على شكل ووفر}
	
	
	يتم إنتاج الخلايا الشمسية من الجيل الأول على رقائق السليكون،هو الأقدم والتكنولوجيا الأكثر شعبية بسبب كفاءة الطاقة العالية ، ويتم تصنيفها إلى نوعين من الخلايا :
	\begin{itemize}
		\item خلايا سيليكونية أحادية البلورة :
		حيت تتكون الخلايا الشمسية من بلورة واحدة من السيليكون ،تقع كفاءتها بين 17\% و18\%
		\item  خلايا سيليكونية متعددة البلورات
		خلايا سيليكونية متعددة البلورات
		تتألف الوحدات الكهروضوئية متعدد البلورات بشكل عام من بلورات مختلفة، إلي جانب بعضها في خلية واحدة،هي حاليا الخلايا الشمسية الأكثر شعبية ،كفاءتها من 12\% إلى 14\%
	\end{itemize}
	\subsubsection{ الجيل الثاني من الخلايا الشمسية (الخلايا الشمسية الرقيقة)}
	
	
	هي أكثر إقتصادا بالمقارنة مع الجيل الأول من الخلايا الشمسية السيليكونية،تحتوي على طبقات إمتصاص خفيفة من سمك 1 ميكرومتر(13)،تتراوح كفاءتها بين 18\%و12\% وتدعي بعض المختبرات الحصول على كفاءة تتجاوز 20\% .
	\subsubsection{ الخلايا الشمسية جيل الثالث-متعددة الوصلات - }
	\begin{itemize}
		
		\item   الخلايا الشمسية  النانوية 
		
		وتعرف كذلك بالخلايا الشمسية ذات النقاط الكمومية (QD) تتميز بالفجوة النطاق القابلة للضبط  و التي تعني أن الطول الموجي القادم من طيف الإشعاع الشمسي الذي يتم إمتصاصه أو نثره يمكن أن يتم الهيمنة عليه والتحكم فيه وضبطه ،فالنقاط الكمومية هي فئة خاصة من أشباه الموصلات تتكون من بلورات نانوية مجموعات دورية من $ II-VI $ أو $ III-V  $أو  $  IIV-VI$  ،وتصل كفاءة تحويل الطاقة في الخلايا الفوتوفولطية المكونة من نقاط كمومية متساوية الأحجام إلى 63\%
		[20،19,21]
		\item  الخلايا الشمسية البوليمرية 
		هذه الخلايا تجمع بين تكنولوجيا أشباه الموصلات وخواص البلاستيك تختلف عن الخلايا الشمسية الإعتيادية القاسية كونها خفيفة الوزن ،وشفافة ومرنة مما يمكن تطبيقها على مساحات كبيرةو تلك أهم مميزات الوجه الحسن للبلاستيك 
		يطلق عليها أيضا الخلايا الكهروضوئية (OPV) وهي تلك التي يتم تصنيعها من مركبات مذابة تشبه الأحبار بحيث يمكن طباعتها على شكل لفات رقيقة من البلاستيك، وتتمتع بمرونة عالية تجعلها قادرة على الإنحناء حول الهياكل، عملية تصنيع هذه الخلايا سهلة وسريعة وغير مكلفة
		[22]
		\item   الخلايا الشمسية الصبغية  
		
		تم تقديم أول خلية من قبل $ Michel Gratzel $ في المعهد السويسري للتكنولولوجيا الفيدرالي ،تستخده هذه الخلايا عموما جزبئات الصبغ بين الأقطاب المختلفة تتكون من أربع مكونات هي قطب أشباه الموصلات   ($ TiO2 $من النوع$   $N و$ NiO $ من النوع $ P  $) ومحسن أصباغ ووسيط للأكسدة وقطب كهربائي مضاد ، تتسم هذه الخلايا بدرجة عالية من المرونة والشفافية ومنخفضة التكلفة أيضا 
		\item  الخلايا الشمسية المركزة 
		
		يتمثل المبدأ الرئيسي للخلايا المركزة في جمع كمية من الطاقة الشمسية على منطقة صغيرة فوق الخلايا الشمسية الكهروضوئية ،وذلك بإستخدام مرايا كبيرة وترتيب العدسات لتركيز أشعة الشمس على منطقة صغيرة على الخلية الشمسية كفاءة هذه الخلايا 40\% [ 19].
		\begin{figure}[h!]
			\centering
			\includegraphics[width=0.5\linewidth, height=0.3\textheight]{"Fig/Fig_I/ CPV"}
			\caption{خلية شمسية مركزة}
			\label{fig:-cpv}
		\end{figure}
		\FloatBarrier			
		
		\item  الخلايا القائمة على البيرفسكات 
		
		هي إكتشاف حدث بين مجتمع الأبحاث الخلايا الشمسية وتمتلك العديد من المزايا على السيليكون التقليدي والخلايا الشمسية القائمة على طبقة رقيقة يمكنها تحقيق كفاءة 31 \%، قد تلعب دورا هاما في بطاريات السيارات الكهربائية من الجيل التالي
		[19]
	\end{itemize}
	\subsection{  مقادير الخلايا الشمسية:  } 
	\begin{itemize}
		\item تيار قصر الدارة $ I_{cc} $:
		\\
		هو التيار الذي يتدفق عبر الوصلة تحت الإضاءة دون إستخدام الجهد، يزداد مع شدة إضاءة الخلية ويعتمد على السطح المضئ،والطول الموجي للإشعاع، وحركة الناقلات ودرجة الحرارة
		\item جهد الدارة المفتوحة  $ V_{oc} $:
		هو الجهد الذي يكون فيه التيار الذي يوفره المولد الكهروضوئي معدوما (الجهد المقاس عندما لايتدفق التيار في الجهاز الكهروضوئي)، ويعتمد أساسا على نوع الخلية الشمسية وإضاءة الخلية وتعطى علاقته بالشكل التالي :
		\begin {equation} 
		V_{oc}=\frac{KT}{q}\log [\frac{I_{ph}}{I_s} +1]
		\end {equation}
		\item معامل الإمتلاء FF
		
		\begin {equation} 
		FF=\frac{P_{max}}{V_ {oc} ×I_ {cc }}
		=\frac{ V_m×I_m}{ V_{oc} ×I{cc}}
		\end {equation}
		تمثل  $  I_m $ و$ V_m  $الكثافة والجهد لكثافة التشغيل مما يسمح بإستخراج الحد الأقصى للإستطاعة  $ P_{max}  $من الخلية \\
		في حال الحد $  FF=1 $ تكون الإستطاعة المستخرجة من الخلية هي الحد الأقصى 
		[23]
		\item  كفاءة تحويل الطاقة الشمسية (المردود)$\eta$:      \\
		وتعرف على أنها القدرة الخارجة من الخلية على القدرة الداخلة إليها (طاقة الشمس )وتعطى بالعلاقة التالية :
		[4]
		\begin {equation} 
		\eta= \frac{ P_{max}}{P_{in}}
		=\frac{ FF×V_{oc}× I_{cc}}{ P_{in}}
		\end {equation}
		$ P_{in} $				           
		إستطاعة الضوء الساقط وتساوي إستطاعة الشمس $ P_{solaire}=100 mw/cm2 $		  
	\end{itemize}
	
	\section { الخلايا الشمسية ذات الآبار الكمومية  }
	لقد حظيت الخلايا الشمسية ذات الآبار الكمومية ، كنهج واعد لتقنية الجيل القادم من الخلايا الكهروضوئية ، بقبول كبير الانتباه في السنوات القليلة الماضية. فتحت التطورات الأخيرة في نمو المواد وهياكل الأجهزة للآبار الكمية سبلًا جديدة لدمج هياكل الآبار الكمية في الجيل التالي من الخلايا الشمسية متعددة الوصلات $ III / V $
	\subsection{تعريف}
	فالخلايا الشمسية ذات الابار الكمية هي هياكل p-i-n حيث هذه الأخيرة تترك فيها طبقة من أشباه الموصلات بين pو  n ، حيث يوجد البئر الكمي  $ QWs $ في الطبقة الداخلية (i) التي تحتوي على هياكل نانوية دورية تتكون من مادة ذات فجوة نطاق منخفضة  محصورة بين حواجز ذات فجوة نطاق أعلى. 
	\\
	حيث نهج $ QW $ الأول هو أنظمة المواد المطابقة للشبكة ، حيث يكون للبئر 
	والحاجز ثوابت شبكية مماثلة للركيزة الأساسية.
	\\
	نهج $ QW $ الثاني هو أنظمة المواد المتوازنة الإجهاد  من خلال موازنة الضغط الانضغاطي الناتج عن البئر مع إجهاد الشد على الحاجز
	\\
	يتم توليد حاملات الشحنة عن طريق التيار المدمج الموجودة في المنطقة $ QW $  وإعادة تركيبها هناك طرقتين لتنقل  حاملات الشحنة -الإلكترونات ، الفجوات -عن طريق الإنجراف : 
	\\
	استخدام الآبار الرقيقة ويتم نقل الحاملات  بواسطة المجال الكهربائي عبر الانبعاث الحراري
	\\
	استخدام حواجز رفيعة ، ويتم نقل المواد الحاملة بواسطة المجال الكهربائي عبر الأنفاق
	
	
	
	\subsection{التحديات والإمكانيات الخلايا ذات الأبار الكمومية }
	القضايا الرئيسية التي يجب معالجتها لتحقيق الكفاءة العالية في الخلايا الشمسية التي تتضمن $ QWs  $هي تمديد عتبة الإمتصاص إلى أطوال موجية أطول ( طاقات أقل )  يمكن إتباع إستراتيجيتين لتحسيين الفجوة 
	\\
	1) زيادة سماكة $ QW $ لتقليل تأثيرات حجم الكم ، 
	\\
	2) زيادة نسبة الإنديوم في $ QW $ .
	\\
	قد يكون لزيادة سمك البئر / محتوى الإنديوم آثار سلبية على أداء الخلية الشمسية $ QW $ ، إذا لم يتم التعامل معها بشكل صحيح.التأثير الأول هو زيادة الضغط الانضغاطي الذي قد يؤدي إلى انتهاك حالة توازن الإجهاد إذا لم يتم زيادة إجهاد الشد لموازنة إجهاد هيكل $ QW $ مما يؤدي إلي تدهور جودة المادة ، التأثير الثاني هو الزيادة في ارتفاع الحاجز الفعال مما يؤثر على حاملات الشحنة بالإضافة إلى تقليل الإنبعاث الحراري 
	\\
	والقضية الثانية التي يجب معالجتها  هي زيادة كفاءة الكم الخارجية $ EQE $ من خلال زيادة عدد الأبار الكمية$ Qw  $الذي يحسن تيار الدارة القصيرة والكفاءة وزيادة سمك $ QW $ الممتص بالنسبة للحاجز 
	[22]
	\begin{figure}[h!]
		\centering
		\includegraphics[width=0.8\linewidth, height=0.5\textheight]{"Fig/Fig_I/ Quantum well"}
		\caption{يوضح انتقال الشحنات في خلية شمسية ذات بئر كمومي}
		\label{fig:-quantum-well}
	\end{figure}
	
	\subsection{ خاتمة }
	قمنا في هذا الفصل بتقديم بعض المفاهيم الأساسية للخلايا الشمسية إبتدا من تحليل ضوء الشمس المكون من عدة أطياف كهرومغناطيسية مرورا بصفة عامة إلى تفاعله مع المادة ثم في المبحث الثاني قدمنا دراسة  حول أشباه النواقل وكيفية الحصول على الوصلة $ pn $ وذكر أليات إنتقال الشحنات في أشباه الموصلات وبعض العمليات التي تتم فيها  ثم في المبحث الثالث قدمنا فيه  مبدأ عملها ومراحل تطويرها من أجل تحسين كفاءتها ثم قدمنا تعريف  للخلايا ذات الأبار الكمومية هذه الأخيرة سنقوم بنمذجتها بإستعمال برنامج Solcore وهو ما سنتطرق إليه في الفصل الثاني .
	